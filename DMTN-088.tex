

\documentclass[DM,authoryear,toc]{lsstdoc}
% lsstdoc documentation: https://lsst-texmf.lsst.io/lsstdoc.html

% Package imports go here.

% Local commands go here.

% To add a short-form title:
% \title[Short title]{Title}
\title{As-is HSC Reprocessing}

% Optional subtitle
% \setDocSubtitle{A subtitle}

\author{%
Hsin-Fang Chiang, Margaret W. G. Johnson
}

\setDocRef{DMTN-088}

\date{\today}

% Optional: name of the document's curator
% \setDocCurator{The Curator of this Document}

\setDocAbstract{%
This document summarizes the status and procedures of the HSC data
reprocessing campaigns done by LDF as of early Fall 2018 cycle.
}

% Change history defined here.
% Order: oldest first.
% Fields: VERSION, DATE, DESCRIPTION, OWNER NAME.
% See LPM-51 for version number policy.
\setDocChangeRecord{%
  \addtohist{1}{YYY-MM-DD}{Unreleased.}{Hsin-Fang Chiang}
}

\begin{document}

% Create the title page.
% Table of contents is added automatically with the "toc" class option.
\maketitle

% ADD CONTENT HERE
In this document, we summarize the current status and operational
processes of the Hyper Suprime-Cam (HSC) data reprocessing campaigns
offered by the LSST Data Facility (LDF) as part of the Batch
Production Services. We describe the service as is in the early
Fall 2018 cycle, but expect the detailed procedures to evolve with
the continuing development and the maturing service.

\section{Goals}

The main goal of reprocessing HSC data using LSST Science Pipelines
is to generate a validation dataset for tests and integration, both
scientifically and operationally, and to regularly scrutinize for
any data or science issues.  Besides allowing to track any quality
or performance changes as the pipeline evolves, the processed dataset
serves as a base set of data products with which the developers can
test a specific component of the science pipelines without the need
to reprocess from raw files, or to test a new tool using state-of-the-art
data products.

Additionally, the reprocessing helps to stress and validate the
infrastructure (both software and hardware) with current science
payload, to provide feedback on operational feasibility of all
current aspects of the system in use, to explore possible operational
strategies, and to mature the policies and procedures of the Batch
Processing Service.

Currently, the HSC reprocessing campaigns are performed at two
scales: (1) the RC scale, and (2) the PDR1 scale; more details are
described in \S \ref{sec:rc} and \S \ref{sec:pdr1}. Recent releases
of the Data Release Production (DRP) pipelines are used. In general,
campaigns are characterized by their goals, inputs (data, calibrations,
codes, and configurations), and cadence.

\section{RC-scale reprocessing}
\label{sec:rc}

The input data are the HSC “RC2” dataset, which includes 3 tracts
of public data and was selected to cover a wide range of data quality
and observing conditions (\jira{DM-11345}). It contains about 8\% of the
full HSC PDR1 dataset (\S \ref{sec:pdr1}).

A calendar-based schedule is used to run a “mini-DRP” with this RC2
dataset once every two weeks. This repeated reprocessing allows
continuous testing and validation of Data Release Production
algorithms.

Even-number weekly stack releases, provided automatically by the
LSST DM Build system and installed by John Swinbank in the designated
software area on the LSST GPFS (\texttt{/software}), are utilized.
To prepare each campaign, the operations staff (1) verifies the
software release was successfully built and installed on GPFS, (2)
verifies the continuous integration package \texttt{ci\_hsc}
can run successfully, (3) evaluates changes to the stack release
compared to the last campaign, and (4) confirms resource availability
on LSST batch cluster and storage. Unless otherwise agreed between
the DRP and LDF teams before launching the campaign, the default
algorithmic configurations are used for the pipelines, as specified
in the config files or source codes in the software stack.

Currently a mini-DRP workflow includes the following top-level pipelines:
\texttt{makeSkyMap.py},\\ \texttt{singleFrameDriver.py}, \texttt{mosaic.py},
\texttt{skyCorrection.py}, \texttt{coaddDriver.py}, and
\texttt{multiBandDriver.py}. Most frontend codes exist in the
\texttt{pipe\_drivers} package and are based on the \texttt{ctrl\_pool}-style
framework. We intend to move away from the \texttt{ctrl\_pool}-style framework
and work towards a production system; however, the transition is
blocked by the Gen3 Butler/Middleware delivery and conversion of
pipelines to the new middleware framework.

To monitor and verify if runs are successful, the job status and
output data products are checked. Failures are characterized and
handled as described in \S \ref{sec:scenarios}.

Generated data products include single frame processing products,
coaddition products, and multiband products, and, upon successful
completion of a campaign, are made available through the LSST
developer infrastructure at NCSA. Management of derived datasets
is done at the X level; individuals files are not managed, as would
be the case in a production system in which records of location,
metadata and provenance are used to centrally manage data. Results
of the four most recent successful runs are retained.  Datasets are
disposed of after they are superseded by the next campaign.

Currently, though temporarily, six QC/QA prototype pipelines from the
\texttt{pipe\_analysis} and \texttt{validate\_drp} packages are included in
the biweekly runs. The \texttt{pipe\_analysis} package is not in the official
stack of \texttt{lsst\_distrib} and was not designed to be included in the
official stack in the first place; however, its outputs are essential
for pipeline development and QA work (DMTN-074) and no replacement
is available yet at time of writing.  Neither \texttt{pipe\_analysis} nor
\texttt{validate\_drp} follow the standard \texttt{pipe\_base}
\texttt{CmdLineTask} framework or the standard Butler usage for data IO.

How the these QC/QA packages will evolve in the long term is a
subject of discussions in DMLT and the QA working group. When the
DM-wide QC/QA plans are clarified, we will run only the QC/QA packages
blessed officially. Before such a plan exists and the packages are
standardized, the LDF staff will continue to work closely with Lauren MacArthur
(\texttt{pipe\_analysis}), Michael Wood-Vasey (\texttt{validate\_drp}), and
Angelo Fausti (\texttt{dispatch\_verify} for uploading metrics to SQuaSH) for
emergent issues.  As QC output products are not as well-defined, the LDF
staff does not track down outputs as diligently as the official products,
and consider the jobs done as long as no fatal errors appear.  In
the short term, adding automatic tests by running \texttt{pipe\_analysis}
in the \texttt{ci\_hsc} package can reduce obvious breakages.

\section{PDR1-scale reprocessing}
\label{sec:pdr1}

On a per request basis and as deemed necessary, LDF reprocesses the
full HSC PDR1 (Public Data Release 1\footnote{\url{https://hsc-release.mtk.nao.ac.jp/}})
dataset with a up-to-date DRP
workflow and a recent software stack. The reprocessing campaign
happens approximately annually. This dataset includes 5654 raw
visits in 7 bands and 3 layers (WIDE, DEEP, UDEEP). This covers 119
tracts in the sky.

With the input \~13 times larger than the RC2 dataset, the PDR1-scale
reprocessing helps identifying edge cases and bugs that do not
appear in the small dataset.  Developers needing more than a few
tracts of data for testing algorithms also use the large processed
dataset. The processed dataset is also potentially useful for PDAC
testing, EPO development, or science verification activities by the
commissioning team.

Software version, configurations, and other processing details that
affect the scientific output products are decided between the DRP
and LDF teams before the start of the run. A RC-scale run with the
same setup precedes the PDR1-scale run to integrate and verify the
components and configurations are nominally ready for bulk processing.
Outputs of the RC-scale run allows QA work and important bug fixes
before effort is spent in the large campaign.  Typically, a few
iterations are involved before the software is finalized and “frozen”
for the processing campaign.

Procedures for monitoring and verifying these larger campaigns are
similar to the biweekly campaigns. During the campaign, some computing
and storage resources are reserved for the campaign use. Completion
of these larger campaigns is announced on the LSST Community forum
for broader consumption beyond the DM team. The output data products
are protected against changes or disasters. Results of the two most
recent successful runs are retained.  Old data products are disposed
before the third campaign starts.

\section{As-is mitigation procedures in irregular scenarios}
\label{sec:scenarios}

Sometimes, things don’t go as smoothly as hoped. Here we describe
some common scenarios and the as-is procedures for handling the
problems.

\begin{itemize}
  \item \textbf{Weekly release is not available in the shared stack
  at /software: wait patiently}

  If a weekly stack is not released or tagged in eups, I wait
  quietly; usually Josh Hoblitt (SQuaRE) would be already aware and
  working on it. If the weekly release has been eups-tagged
  successfully but the installation in the \texttt{/software} shared stack
  fails, I notify John Swinbank.  Currently the LDF team is investigating
  using Singularity containers for the science payload; this would
  break the current dependency on shared stack installation.

  \item \textbf{Failures in building the non-official packages:
  file a JIRA ticket}

  Until 2018-03-31, two packages, \texttt{meas\_mosaic} and
  \texttt{pipe\_analysis},
  needed to be updated and built manually. If they cannot be built,
  for example because the unit tests of \texttt{meas\_mosaic} failed, a
  ticket will be filed so the DRP team can fix it. Meanwhile I start
  the first processing step which does not need \texttt{meas\_mosaic}. Since
  \texttt{w\_2018\_13}, \texttt{meas\_mosaic} has been added to be part of the
  \texttt{lsst\_distrib} weekly release.  So only one package,
  \texttt{pipe\_analysis}, needs special care.

  \item \textbf{Operator user errors: operator’s responsibility}

  The DRP workflow in the current system implies a science workflow
  and there are implicit dependencies in job execution. Currently,
  this level of workflow is taken care of by the operator manually.
  If the implicit workflow is not respected, necessary inputs of a
  job may not be available, so the output products would not be
  produced correctly, possibly without clear errors in the execution
  logs. It’s the operator’s responsibility to shepherd the workflow.
  Other operator errors include specifying output location improperly,
  overwriting files incorrectly, operator-introduced race condition,
  and so on. Some operations require knowledge in Butler and task
  framework implementations so  misunderstanding of the framework
  can lead to mistakes as well. Many of these errors can be minimized
  once a real production system, including file management, is in
  use. The pipeline specification definition is also a design goal
  of the Gen3 Middleware.

  \item \textbf{Transient failures of the processing jobs due to
  hardware issues: LDF handles it}

  Hardware or network glitches such as temporary file system
  unavailability or a faulty node can fail jobs. The operator
  contacts the infrastructure team and creates IHS tickets as needed.
  Once the system is back to stable, the same job will be re-submitted.
  Choosing the operational setup wisely is operator’s responsibility.
  For example, inferior specifications in computing resource need
  can lead to insufficient memory, timeout, job lingering forever,
  or other problems.

  \item \textbf{Reproducible fatal errors from the pipeline: report,
  apply fix if available quickly}

  Occasionally, the weekly stack is broken and does not allow the
  job to finish.  I will verify if the failure is reproducible, and
  report to the Slack channel \#dm-hsc-reprocessing for the DRP team
  to respond.  Sometimes the bug is obvious, such as one from recent
  API changes or new features not caught by the standard CI tools;
  in this case a fix can be available quickly especially with the
  help of the developer who added the new feature.  If a fix is
  available within a day, I will use the fix and re-do the execution.
  If the fix is not available shortly, I may pause the campaign
  (although this hasn’t happened so far; developers are always
  helpful.)

  \item \textbf{Reproducible non-fatal errors from the pipeline:
  carry on without changes}

  Errors that are not in the FATAL-level do not stop the execution.
  Processing continue with the same software. Currently some harmless
  errors appear in the logs and not all errors are carefully vetted.
  Only severe errors that prevent further execution of the pipelines
  stop a processing campaign, despite the outputs may or may not
  be scientifically useful. In the future, the formal QC pipelines
  will be helpful to provide scientific metrics of the data products
  during the campaign execution, and may provide additional feedbacks
  on whether a campaign needs to be paused or specific inputs need
  to be removed.

\end{itemize}


% Include all the relevant bib files.
% https://lsst-texmf.lsst.io/lsstdoc.html#bibliographies
\bibliography{lsst,lsst-dm,refs_ads,refs,books}

\end{document}
