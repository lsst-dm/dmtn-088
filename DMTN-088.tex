

\documentclass[DM,authoryear,toc]{lsstdoc}
% lsstdoc documentation: https://lsst-texmf.lsst.io/lsstdoc.html

% Package imports go here.

% Local commands go here.

% To add a short-form title:
% \title[Short title]{Title}
\title{As-is HSC Reprocessing}

% Optional subtitle
% \setDocSubtitle{A subtitle}

\author{%
Hsin-Fang Chiang, Margaret W. G. Johnson
}

\setDocRef{DMTN-088}

\date{\today}

% Optional: name of the document's curator
% \setDocCurator{The Curator of this Document}

\setDocAbstract{%
This document summarizes the status and procedures of the HSC data
reprocessing campaigns done by LDF as of early Fall 2018 cycle.
}

% Change history defined here.
% Order: oldest first.
% Fields: VERSION, DATE, DESCRIPTION, OWNER NAME.
% See LPM-51 for version number policy.
\setDocChangeRecord{%
  \addtohist{1}{YYY-MM-DD}{Unreleased.}{Hsin-Fang Chiang}
}

\begin{document}

% Create the title page.
% Table of contents is added automatically with the "toc" class option.
\maketitle

% ADD CONTENT HERE
In this document, we summarize the current status and operational
processes of the Hyper Suprime-Cam (HSC) data reprocessing campaigns
offered by the LSST Data Facility (LDF) as part of the Batch
Production Services. We describe the service as is in the early
Fall 2018 cycle, but expect the detailed procedures to evolve with
the continuing development and the maturing service.

\section{Goals}

The main goal of reprocessing HSC data using LSST Science Pipelines
is to generate a validation dataset for tests and integration, both
scientifically and operationally, and to regularly scrutinize for
any data or science issues.  Besides allowing to track any quality
or performance changes as the pipeline evolves, the processed dataset
serves as a base set of data products with which the developers can
test a specific component of the science pipelines without the need
to reprocess from raw files, or to test a new tool using state-of-the-art
data products.

Additionally, the reprocessing helps to stress and validate the
infrastructure (both software and hardware) with current science
payload, to provide feedback on operational feasibility of all
current aspects of the system in use, to explore possible operational
strategies, and to mature the policies and procedures of the Batch
Processing Service.

Currently, the HSC reprocessing campaigns are performed at two
scales: (1) the RC scale, and (2) the PDR1 scale; more details are
described in \S \ref{sec:rc} and \S \ref{sec:pdr1}. Recent releases
of the Data Release Production (DRP) pipelines are used. In general,
campaigns are characterized by their goals, inputs (data, calibrations,
codes, and configurations), and cadence.

\section{RC-scale reprocessing}
\label{sec:rc}

The input data are the HSC “RC2” dataset, which includes 3 tracts
of public data and was selected to cover a wide range of data quality
and observing conditions (\jira{DM-11345}). It contains about 8\% of the
full HSC PDR1 dataset (\S \ref{sec:pdr1}).

A calendar-based schedule is used to run a “mini-DRP” with this RC2
dataset once every two weeks. This repeated reprocessing allows
continuous testing and validation of Data Release Production
algorithms.

Even-number weekly stack releases, provided automatically by the
LSST DM Build system and installed by John Swinbank in the designated
software area on the LSST GPFS (\texttt{/software}), are utilized.
To prepare each campaign, the operations staff (1) verifies the
software release was successfully built and installed on GPFS, (2)
verifies the continuous integration package \texttt{ci\_hsc}
can run successfully, (3) evaluates changes to the stack release
compared to the last campaign, and (4) confirms resource availability
on LSST batch cluster and storage. Unless otherwise agreed between
the DRP and LDF teams before launching the campaign, the default
algorithmic configurations are used for the pipelines, as specified
in the config files or source codes in the software stack.

Currently a mini-DRP workflow includes the following top-level pipelines:
\texttt{makeSkyMap.py},\\ \texttt{singleFrameDriver.py}, \texttt{mosaic.py},
\texttt{skyCorrection.py}, \texttt{coaddDriver.py}, and
\texttt{multiBandDriver.py}. Most frontend codes exist in the
\texttt{pipe\_drivers} package and are based on the \texttt{ctrl\_pool}-style
framework. We intend to move away from the \texttt{ctrl\_pool}-style framework
and work towards a production system; however, the transition is
blocked by the Gen3 Butler/Middleware delivery and conversion of
pipelines to the new middleware framework.

To monitor and verify if runs are successful, the job status and
output data products are checked. Failures are characterized and
handled as described in \S \ref{sec:scenarios}.

Generated data products include single frame processing products,
coaddition products, and multiband products, and, upon successful
completion of a campaign, are made available through the LSST
developer infrastructure at NCSA.  For HSC processing campaigns,
LDF currently manages derived datasets as a site file system;
individuals files are managed by users. When we transition to formal
production processing of LSST data, LDF will be responsible for
managing the location, metadata and provenance records of each file,
and production data will be centrally managed in the Data Backbone.
Results of the four most recent successful runs are retained.  Datasets are
disposed of after they are superseded by the next campaign.

Currently, though temporarily, some QC/QA prototype pipelines from the
\texttt{pipe\_analysis} and \texttt{validate\_drp} packages are included in
the biweekly runs. The \texttt{pipe\_analysis} package is not in the official
stack of \texttt{lsst\_distrib} and was not designed to be included in the
official stack in the first place; however, its outputs are essential
for pipeline development and QA work (\citeds{DMTN-074}) and no replacement
is available yet at time of writing.  Neither \texttt{pipe\_analysis} nor
\texttt{validate\_drp} follow the standard \texttt{pipe\_base}
\texttt{CmdLineTask} framework or the standard Butler usage for data IO.

How the these QC/QA packages will evolve in the long term is a subject of
discussions in DMLT and the QA working group (\citeds{LDM-622}, \jira{RFC-476}).
When the DM-wide QC/QA plans are clarified, we will run only the QC/QA packages
blessed officially. Before such a plan exists and the packages are standardized,
the operations staff will work closely with the specific pipeline owners
(Table \ref{tab:liaison}) for emergent issues, and remove problematic
QC/QA pipelines from the campaign as necessary (also see \S \ref{sec:scenarios}).
As QC output products are not as well-defined, the operations
staff does not track down individual outputs until the official products
are formally defined and documented,
and consider the jobs done as long as no fatal errors appear.  In
the short term, adding automatic CI tests for any QC pipelines,
such as running \texttt{pipe\_analysis}
in the \texttt{ci\_hsc} package, can reduce obvious breakages.

\section{PDR1-scale reprocessing}
\label{sec:pdr1}

On a per request basis and as deemed necessary, LDF reprocesses the
full HSC PDR1 (Public Data Release 1\footnote{\url{https://hsc-release.mtk.nao.ac.jp/}})
dataset with a up-to-date DRP
workflow and a recent software stack. The reprocessing campaign
happens approximately annually. This dataset includes 5654 raw
visits in 7 bands and 3 layers (WIDE, DEEP, UDEEP). This covers 119
tracts in the sky.

With the input \~13 times larger than the RC2 dataset, the PDR1-scale
reprocessing helps identifying edge cases and bugs that do not
appear in the small dataset.  Developers needing more than a few
tracts of data for testing algorithms also use the large processed
dataset. The processed dataset is also potentially useful for PDAC
testing, EPO development, or science verification activities by the
commissioning team.

Software version, configurations, and other processing details that
affect the scientific output products are decided between the DRP
and LDF teams before the start of the run. A RC-scale run with the
same setup precedes the PDR1-scale run to integrate and verify the
components and configurations are nominally ready for bulk processing.
Outputs of the RC-scale run allows QA work and important bug fixes
before effort is spent in the large campaign.  Typically, a few
iterations are involved before the software is finalized and “frozen”
for the processing campaign.

Procedures for monitoring and verifying these larger campaigns are
similar to the biweekly campaigns. During the campaign, some computing
and storage resources are reserved for the campaign use. Completion
of these larger campaigns is announced on the LSST Community forum
for broader consumption beyond the DM team. The output data products
are protected against changes or disasters. Results of the two most
recent successful runs are retained.  Old data products are disposed
before the third campaign starts.

\section{As-is mitigation procedures in irregular scenarios}
\label{sec:scenarios}

Sometimes, things don’t go as smoothly as hoped. Here we describe
some common scenarios and the as-is procedures for handling the
problems.

\begin{itemize}
  \item \textbf{Weekly release is not available in the shared stack
  at /software: wait patiently}

  If a weekly stack is not released or tagged in eups, I wait
  quietly; usually Josh Hoblitt (SQuaRE) would be already aware and
  working on it. If the weekly release has been eups-tagged
  successfully but the installation in the \texttt{/software} shared stack
  fails, I notify John Swinbank.  Currently the LDF team is investigating
  using Singularity containers for the science payload; this would
  break the current dependency on shared stack installation.

  \item \textbf{Failures in building the non-official packages:
  file a JIRA ticket}

  Until 2018-03-31, two packages, \texttt{meas\_mosaic} and
  \texttt{pipe\_analysis},
  needed to be updated and built manually. If they cannot be built,
  for example because the unit tests of \texttt{meas\_mosaic} failed, a
  ticket will be filed so the DRP team can fix it. Meanwhile the first
  processing step which does not need \texttt{meas\_mosaic} could start. Since
  \texttt{w\_2018\_13}, \texttt{meas\_mosaic} has been added to be part of the
  \texttt{lsst\_distrib} weekly release.  So only one package,
  \texttt{pipe\_analysis}, needs special care.

  \item \textbf{Operator user errors: operator’s responsibility}

  The DRP workflow in the current system implies a science workflow
  and there are implicit dependencies in job execution. Currently,
  this level of workflow is taken care of by the operator manually.
  If the implicit workflow is not respected, necessary inputs of a
  job may not be available, so the output products would not be
  produced correctly, possibly without clear errors in the execution
  logs. It’s the operator’s responsibility to shepherd the workflow.
  Other operator errors include specifying output location improperly,
  overwriting files incorrectly, operator-introduced race condition,
  and so on. Some operations require knowledge in Butler and task
  framework implementations so  misunderstanding of the framework
  can lead to mistakes as well. Many of these errors can be minimized
  once a real production system, including file management, is in
  use. The pipeline specification definition is also a design goal
  of the Gen3 Middleware.

  \item \textbf{Transient failures of the processing jobs due to
  hardware issues: LDF handles it}

  Hardware or network glitches such as temporary file system
  unavailability or a faulty node can fail jobs. The operator
  contacts the infrastructure team and creates IHS tickets as needed.
  Once the system is back to stable, the same job will be re-submitted.
  Choosing the operational setup wisely is operator’s responsibility.
  For example, inferior specifications in computing resource need
  can lead to insufficient memory, timeout, job lingering forever,
  or other problems.

  \item \textbf{Reproducible fatal errors from the pipeline: report,
  apply fix if available quickly}

  Occasionally, the weekly stack is broken and does not allow the job to
  finish. The operations staff will verify if the failure is reproducible, and
  report to the Slack channel \#dm-hsc-reprocessing for the DRP team
  or the specific pipeline owners (Table \ref{tab:liaison})
  to respond. Jira tickets containing the error messages will be
  filed, and the operations staff can then choose to abort the
  part of the campaign that depend on the data from the failed
  pipelines.  Sometimes the bug is obvious, such as one from recent
  API changes or new features not caught by the standard CI tools;
  in this case a fix can be available quickly especially with the
  help of the developer who added the new feature.  If a fix is
  available within a day, the operations staff may continue the
  execution using a new software version with the fix applied.  If
  the fix is not available shortly, the operations staff may pause
  or cancel the campaign.
  (At the time of writing, no biweekly campaign has been aborted
  so far despite the occasional fatal bugs; developers have been
  helpful.)

  \item \textbf{Reproducible non-fatal errors from the pipeline:
  carry on without changes}

  Errors that are not in the FATAL-level do not stop the execution.
  Processing continue with the same software. Currently some harmless
  errors appear in the logs and not all errors are carefully vetted.
  Only severe errors that prevent further execution of the pipelines
  stop a processing campaign, despite the outputs may or may not
  be scientifically useful. In the future, the formal QC pipelines
  will be helpful to provide scientific metrics of the data products
  during the campaign execution, and may provide additional feedbacks
  on whether a campaign needs to be paused or specific inputs need
  to be removed. Currently, only QC metrics from the single frame
  processing are uploaded to the SQuaSH dashboard \footnote{The metrics
  are uploaded to \url{https://squash.lsst.codes} using the
  \texttt{dispatch\_verify.py} tool as listed in Table \ref{tab:liaison}.};
  further QC metrics will be calculated and uploaded once the responsible
  pipelines are ready.

\end{itemize}

\appendix

\section{Current Science Pipelines Payload}

In Table \ref{tab:liaison}, we list the pipelines currently included
in the biweekly campaigns (\S \ref{sec:rc}) and their corresponding
contacts. The contact may delegate the debugging responsibility to
other team members.
\begin{table}[h]
  \caption{Pipelines currently included in biweekly RC reprocessing}
  \label{tab:liaison}
  \begin{center}
  \begin{tabular}{ccc}
    \hline\hline
    \textbf{package}  &
    \textbf{pipeline} &
    \textbf{contact}  \\
    \hline\hline
    pipe\_tasks    & \texttt{makeSkyMap.py}           & Jim Bosch \\
    pipe\_drivers  & \texttt{singleFrameDriver.py}    & Jim Bosch \\
    pipe\_drivers  & \texttt{skyCorrection.py}        & Jim Bosch \\
    meas\_mosaic   & \texttt{mosaic.py}               & Jim Bosch \\
    pipe\_drivers  & \texttt{coaddDriver.py}          & Jim Bosch \\
    pipe\_drivers  & \texttt{multiBandDriver.py}      & Jim Bosch \\
    pipe\_analysis & \texttt{visitAnalysis.py}        & Lauren MacArthur \\
    pipe\_analysis & \texttt{coaddAnalysis.py}        & Lauren MacArthur \\
    pipe\_analysis & \texttt{colorAnalysis.py}        & Lauren MacArthur \\
    pipe\_analysis & \texttt{compareVisitAnalysis.py} & Lauren MacArthur \\
    pipe\_analysis & \texttt{compareCoaddAnalysis.py} & Lauren MacArthur \\
    validate\_drp  & \texttt{matchedVisitMetrics.py}  & Michael Wood-Vasey \\
    validate\_drp  & \texttt{validateDrp.py}          & Michael Wood-Vasey \\
    validate\_drp  & \texttt{reportPerformance.py}    & Michael Wood-Vasey \\
    verify         & \texttt{dispatch\_verify.py}     & Angelo Fausti \\
    \hline\hline
  \end{tabular}
  \end{center}
\end{table}


\section{Stack environment variables}
An example of the environment variables set up for running a HSC RC reprocessing is below.
\begin{verbatim}
AFW_DIR=/software/lsstsw/stack3_20171023/stack/miniconda3-4.3.21-10a4fa6/Linux64/afw/16.0-31-gd4f695684
APR_DIR=/software/lsstsw/stack3_20171023/stack/miniconda3-4.3.21-10a4fa6/Linux64/apr/1.5.2
APR_UTIL_DIR=/software/lsstsw/stack3_20171023/stack/miniconda3-4.3.21-10a4fa6/Linux64/apr_util/1.5.4
ASTROMETRY_NET_DATA_DIR=/software/lsstsw/stack3_20171023/stack/miniconda3-4.3.21-10a4fa6/Linux64/astrometry_net_data/8.0.0.0+40
ASTROMETRY_NET_DIR=/software/lsstsw/stack3_20171023/stack/miniconda3-4.3.21-10a4fa6/Linux64/astrometry_net/0.67.123ff3e.lsst1+8
ASTROPY_DIR=/software/lsstsw/stack3_20171023/stack/miniconda3-4.3.21-10a4fa6/Linux64/astropy/2.0.1+2
ASTSHIM_DIR=/software/lsstsw/stack3_20171023/stack/miniconda3-4.3.21-10a4fa6/Linux64/astshim/16.0-2-g0febb12+5
BASE_DIR=/software/lsstsw/stack3_20171023/stack/miniconda3-4.3.21-10a4fa6/Linux64/base/16.0-3-g8e51203
BOOST_DIR=/software/lsstsw/stack3_20171023/stack/miniconda3-4.3.21-10a4fa6/Linux64/boost/1.68
CFITSIO_DIR=/software/lsstsw/stack3_20171023/stack/miniconda3-4.3.21-10a4fa6/Linux64/cfitsio/3360.lsst5
CMAKE_PREFIX_PATH=/software/lsstsw/stack3_20171023/stack/miniconda3-4.3.21-10a4fa6/Linux64/pybind11/2.2.3.lsst1
COADD_CHISQUARED_DIR=/software/lsstsw/stack3_20171023/stack/miniconda3-4.3.21-10a4fa6/Linux64/coadd_chisquared/16.0-2-g839ba83+22
COADD_UTILS_DIR=/software/lsstsw/stack3_20171023/stack/miniconda3-4.3.21-10a4fa6/Linux64/coadd_utils/16.0-4-g8a0f11a+6
CP_PIPE_DIR=/software/lsstsw/stack3_20171023/stack/miniconda3-4.3.21-10a4fa6/Linux64/cp_pipe/16.0+32
CTRL_EXECUTE_DIR=/software/lsstsw/stack3_20171023/stack/miniconda3-4.3.21-10a4fa6/Linux64/ctrl_execute/15.0+34
CTRL_ORCA_DIR=/software/lsstsw/stack3_20171023/stack/miniconda3-4.3.21-10a4fa6/Linux64/ctrl_orca/16.0-1-g9633d82+9
CTRL_PLATFORM_LSSTVC_DIR=/software/lsstsw/stack3_20171023/stack/miniconda3-4.3.21-10a4fa6/Linux64/ctrl_platform_lsstvc/15.0+34
CTRL_POOL_DIR=/software/lsstsw/stack3_20171023/stack/miniconda3-4.3.21-10a4fa6/Linux64/ctrl_pool/16.0-1-g14407aa+17
DAF_BASE_DIR=/software/lsstsw/stack3_20171023/stack/miniconda3-4.3.21-10a4fa6/Linux64/daf_base/16.0-4-g50d071e+6
DAF_BUTLER_DIR=/software/lsstsw/stack3_20171023/stack/miniconda3-4.3.21-10a4fa6/Linux64/daf_butler/master-gde6ace3019
DAF_PERSISTENCE_DIR=/software/lsstsw/stack3_20171023/stack/miniconda3-4.3.21-10a4fa6/Linux64/daf_persistence/16.0-3-g3806c63+6
DATASET=HSC RC2
DATASET_REPO_URL=https://jira.lsstcorp.org/browse/DM-11345
DISPLAY=localhost:17.0
DISPLAY_DS9_DIR=/software/lsstsw/stack3_20171023/stack/miniconda3-4.3.21-10a4fa6/Linux64/display_ds9/16.0-2-g9d5294e+16
DISPLAY_FIREFLY_DIR=/software/lsstsw/stack3_20171023/stack/miniconda3-4.3.21-10a4fa6/Linux64/display_firefly/16.0-4-g03cf288+5
DISPLAY_MATPLOTLIB_DIR=/software/lsstsw/stack3_20171023/stack/miniconda3-4.3.21-10a4fa6/Linux64/display_matplotlib/16.0-3-gcfd6c53+14
DOXYGEN_DIR=/software/lsstsw/stack3_20171023/stack/miniconda3-4.3.21-10a4fa6/Linux64/doxygen/1.8.13.lsst2
DYLD_LIBRARY_PATH=/software/lsstsw/stack3_20171023/stack/miniconda3-4.3.21-10a4fa6/Linux64/meas_mosaic/16.0-10-g3775aa7+2/lib:/software/lsstsw/stack3_20171023/stack/miniconda3-4.3.21-10a4fa6/Linux64/cp_pipe/16.0+32/lib:/software/lsstsw/stack3_20171023/stack/miniconda3-4.3.21-10a4fa6/Linux64/jointcal_cholmod/master-g48cb81d145+44/lib:/software/lsstsw/stack3_20171023/stack/miniconda3-4.3.21-10a4fa6/Linux64/jointcal/16.0-15-g8e16a51+1/lib:/software/lsstsw/stack3_20171023/stack/miniconda3-4.3.21-10a4fa6/Linux64/obs_subaru/16.0-20-gb58d072b/lib:/software/lsstsw/stack3_20171023/stack/miniconda3-4.3.21-10a4fa6/Linux64/mpich/3.2.1/lib:/software/lsstsw/stack3_20171023/stack/miniconda3-4.3.21-10a4fa6/Linux64/tmv/0.73.lsst2+2/lib:/software/lsstsw/stack3_20171023/stack/miniconda3-4.3.21-10a4fa6/Linux64/galsim/1.6.0.lsst2-1-gd5e7736+1/lib:/software/lsstsw/stack3_20171023/stack/miniconda3-4.3.21-10a4fa6/Linux64/meas_extensions_shapeHSM/16.0-5-g3a22acd+6/lib:/software/lsstsw/stack3_20171023/stack/miniconda3-4.3.21-10a4fa6/Linux64/meas_extensions_photometryKron/16.0-4-ga5d8928+6/lib:/software/lsstsw/stack3_20171023/stack/miniconda3-4.3.21-10a4fa6/Linux64/psfex/16.0+4/lib:/software/lsstsw/stack3_20171023/stack/miniconda3-4.3.21-10a4fa6/Linux64/meas_extensions_psfex/16.0-8-gc315727+6/lib:/software/lsstsw/stack3_20171023/stack/miniconda3-4.3.21-10a4fa6/Linux64/xpa/2.1.15.lsst3/lib:/software/lsstsw/stack3_20171023/stack/miniconda3-4.3.21-10a4fa6/Linux64/meas_extensions_simpleShape/16.0-5-gb3f8a4b+16/lib:/software/lsstsw/stack3_20171023/stack/miniconda3-4.3.21-10a4fa6/Linux64/healpy/1.10.3.lsst1+6/lib:/software/lsstsw/stack3_20171023/stack/miniconda3-4.3.21-10a4fa6/Linux64/coadd_chisquared/16.0-2-g839ba83+22/lib:/software/lsstsw/stack3_20171023/stack/miniconda3-4.3.21-10a4fa6/Linux64/ip_diffim/16.0-8-g4aca173+17/lib:/software/lsstsw/stack3_20171023/stack/miniconda3-4.3.21-10a4fa6/Linux64/ip_isr/16.0-7-g2c7e7ec+8/lib:/software/lsstsw/stack3_20171023/stack/miniconda3-4.3.21-10a4fa6/Linux64/meas_extensions_astrometryNet/16.0-4-g347813e+16/lib:/software/lsstsw/stack3_20171023/stack/miniconda3-4.3.21-10a4fa6/Linux64/wcslib/5.13.lsst1+2/lib:/software/lsstsw/stack3_20171023/stack/miniconda3-4.3.21-10a4fa6/Linux64/astrometry_net/0.67.123ff3e.lsst1+8/lib:/software/lsstsw/stack3_20171023/stack/miniconda3-4.3.21-10a4fa6/Linux64/meas_astrom/16.0-10-g7547e25+2/lib:/software/lsstsw/stack3_20171023/stack/miniconda3-4.3.21-10a4fa6/Linux64/shapelet/16.0-6-gf0acd13+8/lib:/software/lsstsw/stack3_20171023/stack/miniconda3-4.3.21-10a4fa6/Linux64/meas_modelfit/16.0-11-gce733cf+7/lib:/software/lsstsw/stack3_20171023/stack/miniconda3-4.3.21-10a4fa6/Linux64/coadd_utils/16.0-4-g8a0f11a+6/lib:/software/lsstsw/stack3_20171023/stack/miniconda3-4.3.21-10a4fa6/Linux64/meas_base/16.0-11-gf9130ea+15/lib:/software/lsstsw/stack3_20171023/stack/miniconda3-4.3.21-10a4fa6/Linux64/meas_algorithms/16.0-13-g0fce7516+2/lib:/software/lsstsw/stack3_20171023/stack/miniconda3-4.3.21-10a4fa6/Linux64/starlink_ast/lsst-dev-gd6cc4e835a/lib:/software/lsstsw/stack3_20171023/stack/miniconda3-4.3.21-10a4fa6/Linux64/astshim/16.0-2-g0febb12+5/lib:/software/lsstsw/stack3_20171023/stack/miniconda3-4.3.21-10a4fa6/Linux64/gsl/2.4/lib:/software/lsstsw/stack3_20171023/stack/miniconda3-4.3.21-10a4fa6/Linux64/minuit2/5.34.14/lib:/software/lsstsw/stack3_20171023/stack/miniconda3-4.3.21-10a4fa6/Linux64/cfitsio/3360.lsst5/lib:/software/lsstsw/stack3_20171023/stack/miniconda3-4.3.21-10a4fa6/Linux64/sphgeom/16.0-4-g5f3a788+5/lib:/software/lsstsw/stack3_20171023/stack/miniconda3-4.3.21-10a4fa6/Linux64/fftw/3.3.4.lsst2/lib:/software/lsstsw/stack3_20171023/stack/miniconda3-4.3.21-10a4fa6/Linux64/geom/16.0-8-g80699e5+1/lib:/software/lsstsw/stack3_20171023/stack/miniconda3-4.3.21-10a4fa6/Linux64/pex_config/16.0-3-g9645794+3/lib:/software/lsstsw/stack3_20171023/stack/miniconda3-4.3.21-10a4fa6/Linux64/libyaml/0.1.7/lib:/software/lsstsw/stack3_20171023/stack/miniconda3-4.3.21-10a4fa6/Linux64/pex_policy/16.0-2-gf41ba6b+3/lib:/software/lsstsw/stack3_20171023/stack/miniconda3-4.3.21-10a4fa6/Linux64/apr_util/1.5.4/lib:/software/lsstsw/stack3_20171023/stack/miniconda3-4.3.21-10a4fa6/Linux64/apr/1.5.2/lib:/software/lsstsw/stack3_20171023/stack/miniconda3-4.3.21-10a4fa6/Linux64/log4cxx/0.10.0.lsst7/lib:/software/lsstsw/stack3_20171023/stack/miniconda3-4.3.21-10a4fa6/Linux64/log/16.0-1-gce273f5+6/lib:/software/lsstsw/stack3_20171023/stack/miniconda3-4.3.21-10a4fa6/Linux64/daf_persistence/16.0-3-g3806c63+6/lib:/software/lsstsw/stack3_20171023/stack/miniconda3-4.3.21-10a4fa6/Linux64/boost/1.68/lib:/software/lsstsw/stack3_20171023/stack/miniconda3-4.3.21-10a4fa6/Linux64/pex_exceptions/16.0-2-gf4e7cdd+3/lib:/software/lsstsw/stack3_20171023/stack/miniconda3-4.3.21-10a4fa6/Linux64/base/16.0-3-g8e51203/lib:/software/lsstsw/stack3_20171023/stack/miniconda3-4.3.21-10a4fa6/Linux64/utils/16.0-6-g3610b4f+3/lib:/software/lsstsw/stack3_20171023/stack/miniconda3-4.3.21-10a4fa6/Linux64/daf_base/16.0-4-g50d071e+6/lib:/software/lsstsw/stack3_20171023/stack/miniconda3-4.3.21-10a4fa6/Linux64/afw/16.0-31-gd4f695684/lib:/software/lsstsw/stack3_20171023/stack/miniconda3-4.3.21-10a4fa6/Linux64/meas_deblender/16.0-7-g37292d5+6/lib
EIGEN_DIR=/software/lsstsw/stack3_20171023/stack/miniconda3-4.3.21-10a4fa6/Linux64/eigen/3.3.4.lsst1
ESUTIL_DIR=/software/lsstsw/stack3_20171023/stack/miniconda3-4.3.21-10a4fa6/Linux64/esutil/0.6.2.5.lsst1+2
EUPS_DIR=/software/lsstsw/stack/eups/2.1.4
EUPS_PATH=/software/lsstsw/stack3_20171023/stack/miniconda3-4.3.21-10a4fa6
EUPS_PKGROOT=https://eups.lsst.codes/stack/src
EUPS_SHELL=sh
FFTW_DIR=/software/lsstsw/stack3_20171023/stack/miniconda3-4.3.21-10a4fa6/Linux64/fftw/3.3.4.lsst2
FIREFLY_CLIENT_DIR=/software/lsstsw/stack3_20171023/stack/miniconda3-4.3.21-10a4fa6/Linux64/firefly_client/lsst-dev-gd3d76961fa
FLAKE8_DIR=/software/lsstsw/stack3_20171023/stack/miniconda3-4.3.21-10a4fa6/Linux64/flake8/3.5.0+8
GALSIM_DIR=/software/lsstsw/stack3_20171023/stack/miniconda3-4.3.21-10a4fa6/Linux64/galsim/1.6.0.lsst2-1-gd5e7736+1
GEOM_DIR=/software/lsstsw/stack3_20171023/stack/miniconda3-4.3.21-10a4fa6/Linux64/geom/16.0-8-g80699e5+1
GSL_DIR=/software/lsstsw/stack3_20171023/stack/miniconda3-4.3.21-10a4fa6/Linux64/gsl/2.4
HEALPY_DIR=/software/lsstsw/stack3_20171023/stack/miniconda3-4.3.21-10a4fa6/Linux64/healpy/1.10.3.lsst1+6
HISTCONTROL=ignoredups
HISTSIZE=1000
INFOPATH=/opt/rh/devtoolset-6/root/usr/share/info
IP_DIFFIM_DIR=/software/lsstsw/stack3_20171023/stack/miniconda3-4.3.21-10a4fa6/Linux64/ip_diffim/16.0-8-g4aca173+17
IP_ISR_DIR=/software/lsstsw/stack3_20171023/stack/miniconda3-4.3.21-10a4fa6/Linux64/ip_isr/16.0-7-g2c7e7ec+8
JOINTCAL_CHOLMOD_DIR=/software/lsstsw/stack3_20171023/stack/miniconda3-4.3.21-10a4fa6/Linux64/jointcal_cholmod/master-g48cb81d145+44
JOINTCAL_DIR=/software/lsstsw/stack3_20171023/stack/miniconda3-4.3.21-10a4fa6/Linux64/jointcal/16.0-15-g8e16a51+1
LANG=en_US.UTF-8
LD_LIBRARY_PATH=/software/lsstsw/stack3_20171023/stack/miniconda3-4.3.21-10a4fa6/Linux64/meas_mosaic/16.0-10-g3775aa7+2/lib:/software/lsstsw/stack3_20171023/stack/miniconda3-4.3.21-10a4fa6/Linux64/cp_pipe/16.0+32/lib:/software/lsstsw/stack3_20171023/stack/miniconda3-4.3.21-10a4fa6/Linux64/jointcal_cholmod/master-g48cb81d145+44/lib:/software/lsstsw/stack3_20171023/stack/miniconda3-4.3.21-10a4fa6/Linux64/jointcal/16.0-15-g8e16a51+1/lib:/software/lsstsw/stack3_20171023/stack/miniconda3-4.3.21-10a4fa6/Linux64/obs_subaru/16.0-20-gb58d072b/lib:/software/lsstsw/stack3_20171023/stack/miniconda3-4.3.21-10a4fa6/Linux64/mpich/3.2.1/lib:/software/lsstsw/stack3_20171023/stack/miniconda3-4.3.21-10a4fa6/Linux64/tmv/0.73.lsst2+2/lib:/software/lsstsw/stack3_20171023/stack/miniconda3-4.3.21-10a4fa6/Linux64/galsim/1.6.0.lsst2-1-gd5e7736+1/lib:/software/lsstsw/stack3_20171023/stack/miniconda3-4.3.21-10a4fa6/Linux64/meas_extensions_shapeHSM/16.0-5-g3a22acd+6/lib:/software/lsstsw/stack3_20171023/stack/miniconda3-4.3.21-10a4fa6/Linux64/meas_extensions_photometryKron/16.0-4-ga5d8928+6/lib:/software/lsstsw/stack3_20171023/stack/miniconda3-4.3.21-10a4fa6/Linux64/psfex/16.0+4/lib:/software/lsstsw/stack3_20171023/stack/miniconda3-4.3.21-10a4fa6/Linux64/meas_extensions_psfex/16.0-8-gc315727+6/lib:/software/lsstsw/stack3_20171023/stack/miniconda3-4.3.21-10a4fa6/Linux64/xpa/2.1.15.lsst3/lib:/software/lsstsw/stack3_20171023/stack/miniconda3-4.3.21-10a4fa6/Linux64/meas_extensions_simpleShape/16.0-5-gb3f8a4b+16/lib:/software/lsstsw/stack3_20171023/stack/miniconda3-4.3.21-10a4fa6/Linux64/healpy/1.10.3.lsst1+6/lib:/software/lsstsw/stack3_20171023/stack/miniconda3-4.3.21-10a4fa6/Linux64/coadd_chisquared/16.0-2-g839ba83+22/lib:/software/lsstsw/stack3_20171023/stack/miniconda3-4.3.21-10a4fa6/Linux64/ip_diffim/16.0-8-g4aca173+17/lib:/software/lsstsw/stack3_20171023/stack/miniconda3-4.3.21-10a4fa6/Linux64/ip_isr/16.0-7-g2c7e7ec+8/lib:/software/lsstsw/stack3_20171023/stack/miniconda3-4.3.21-10a4fa6/Linux64/meas_extensions_astrometryNet/16.0-4-g347813e+16/lib:/software/lsstsw/stack3_20171023/stack/miniconda3-4.3.21-10a4fa6/Linux64/wcslib/5.13.lsst1+2/lib:/software/lsstsw/stack3_20171023/stack/miniconda3-4.3.21-10a4fa6/Linux64/astrometry_net/0.67.123ff3e.lsst1+8/lib:/software/lsstsw/stack3_20171023/stack/miniconda3-4.3.21-10a4fa6/Linux64/meas_astrom/16.0-10-g7547e25+2/lib:/software/lsstsw/stack3_20171023/stack/miniconda3-4.3.21-10a4fa6/Linux64/shapelet/16.0-6-gf0acd13+8/lib:/software/lsstsw/stack3_20171023/stack/miniconda3-4.3.21-10a4fa6/Linux64/meas_modelfit/16.0-11-gce733cf+7/lib:/software/lsstsw/stack3_20171023/stack/miniconda3-4.3.21-10a4fa6/Linux64/coadd_utils/16.0-4-g8a0f11a+6/lib:/software/lsstsw/stack3_20171023/stack/miniconda3-4.3.21-10a4fa6/Linux64/meas_base/16.0-11-gf9130ea+15/lib:/software/lsstsw/stack3_20171023/stack/miniconda3-4.3.21-10a4fa6/Linux64/meas_algorithms/16.0-13-g0fce7516+2/lib:/software/lsstsw/stack3_20171023/stack/miniconda3-4.3.21-10a4fa6/Linux64/starlink_ast/lsst-dev-gd6cc4e835a/lib:/software/lsstsw/stack3_20171023/stack/miniconda3-4.3.21-10a4fa6/Linux64/astshim/16.0-2-g0febb12+5/lib:/software/lsstsw/stack3_20171023/stack/miniconda3-4.3.21-10a4fa6/Linux64/gsl/2.4/lib:/software/lsstsw/stack3_20171023/stack/miniconda3-4.3.21-10a4fa6/Linux64/minuit2/5.34.14/lib:/software/lsstsw/stack3_20171023/stack/miniconda3-4.3.21-10a4fa6/Linux64/cfitsio/3360.lsst5/lib:/software/lsstsw/stack3_20171023/stack/miniconda3-4.3.21-10a4fa6/Linux64/sphgeom/16.0-4-g5f3a788+5/lib:/software/lsstsw/stack3_20171023/stack/miniconda3-4.3.21-10a4fa6/Linux64/fftw/3.3.4.lsst2/lib:/software/lsstsw/stack3_20171023/stack/miniconda3-4.3.21-10a4fa6/Linux64/geom/16.0-8-g80699e5+1/lib:/software/lsstsw/stack3_20171023/stack/miniconda3-4.3.21-10a4fa6/Linux64/pex_config/16.0-3-g9645794+3/lib:/software/lsstsw/stack3_20171023/stack/miniconda3-4.3.21-10a4fa6/Linux64/libyaml/0.1.7/lib:/software/lsstsw/stack3_20171023/stack/miniconda3-4.3.21-10a4fa6/Linux64/pex_policy/16.0-2-gf41ba6b+3/lib:/software/lsstsw/stack3_20171023/stack/miniconda3-4.3.21-10a4fa6/Linux64/apr_util/1.5.4/lib:/software/lsstsw/stack3_20171023/stack/miniconda3-4.3.21-10a4fa6/Linux64/apr/1.5.2/lib:/software/lsstsw/stack3_20171023/stack/miniconda3-4.3.21-10a4fa6/Linux64/log4cxx/0.10.0.lsst7/lib:/software/lsstsw/stack3_20171023/stack/miniconda3-4.3.21-10a4fa6/Linux64/log/16.0-1-gce273f5+6/lib:/software/lsstsw/stack3_20171023/stack/miniconda3-4.3.21-10a4fa6/Linux64/daf_persistence/16.0-3-g3806c63+6/lib:/software/lsstsw/stack3_20171023/stack/miniconda3-4.3.21-10a4fa6/Linux64/boost/1.68/lib:/software/lsstsw/stack3_20171023/stack/miniconda3-4.3.21-10a4fa6/Linux64/pex_exceptions/16.0-2-gf4e7cdd+3/lib:/software/lsstsw/stack3_20171023/stack/miniconda3-4.3.21-10a4fa6/Linux64/base/16.0-3-g8e51203/lib:/software/lsstsw/stack3_20171023/stack/miniconda3-4.3.21-10a4fa6/Linux64/utils/16.0-6-g3610b4f+3/lib:/software/lsstsw/stack3_20171023/stack/miniconda3-4.3.21-10a4fa6/Linux64/daf_base/16.0-4-g50d071e+6/lib:/software/lsstsw/stack3_20171023/stack/miniconda3-4.3.21-10a4fa6/Linux64/afw/16.0-31-gd4f695684/lib:/software/lsstsw/stack3_20171023/stack/miniconda3-4.3.21-10a4fa6/Linux64/meas_deblender/16.0-7-g37292d5+6/lib:/opt/rh/devtoolset-6/root/usr/lib64:/opt/rh/devtoolset-6/root/usr/lib
LESSOPEN=||/usr/bin/lesspipe.sh %s
LIBYAML_DIR=/software/lsstsw/stack3_20171023/stack/miniconda3-4.3.21-10a4fa6/Linux64/libyaml/0.1.7
LMFIT_DIR=/software/lsstsw/stack3_20171023/stack/miniconda3-4.3.21-10a4fa6/Linux64/lmfit/0.9.3+8
LOG4CXX_DIR=/software/lsstsw/stack3_20171023/stack/miniconda3-4.3.21-10a4fa6/Linux64/log4cxx/0.10.0.lsst7
LOG_DIR=/software/lsstsw/stack3_20171023/stack/miniconda3-4.3.21-10a4fa6/Linux64/log/16.0-1-gce273f5+6
LSST_APPS_DIR=/software/lsstsw/stack3_20171023/stack/miniconda3-4.3.21-10a4fa6/Linux64/lsst_apps/14.0+178
LSST_DISTRIB_DIR=/software/lsstsw/stack3_20171023/stack/miniconda3-4.3.21-10a4fa6/Linux64/lsst_distrib/16.0+41
LSST_LIBRARY_PATH=/software/lsstsw/stack3_20171023/stack/miniconda3-4.3.21-10a4fa6/Linux64/meas_mosaic/16.0-10-g3775aa7+2/lib:/software/lsstsw/stack3_20171023/stack/miniconda3-4.3.21-10a4fa6/Linux64/cp_pipe/16.0+32/lib:/software/lsstsw/stack3_20171023/stack/miniconda3-4.3.21-10a4fa6/Linux64/jointcal_cholmod/master-g48cb81d145+44/lib:/software/lsstsw/stack3_20171023/stack/miniconda3-4.3.21-10a4fa6/Linux64/jointcal/16.0-15-g8e16a51+1/lib:/software/lsstsw/stack3_20171023/stack/miniconda3-4.3.21-10a4fa6/Linux64/obs_subaru/16.0-20-gb58d072b/lib:/software/lsstsw/stack3_20171023/stack/miniconda3-4.3.21-10a4fa6/Linux64/mpich/3.2.1/lib:/software/lsstsw/stack3_20171023/stack/miniconda3-4.3.21-10a4fa6/Linux64/tmv/0.73.lsst2+2/lib:/software/lsstsw/stack3_20171023/stack/miniconda3-4.3.21-10a4fa6/Linux64/galsim/1.6.0.lsst2-1-gd5e7736+1/lib:/software/lsstsw/stack3_20171023/stack/miniconda3-4.3.21-10a4fa6/Linux64/meas_extensions_shapeHSM/16.0-5-g3a22acd+6/lib:/software/lsstsw/stack3_20171023/stack/miniconda3-4.3.21-10a4fa6/Linux64/meas_extensions_photometryKron/16.0-4-ga5d8928+6/lib:/software/lsstsw/stack3_20171023/stack/miniconda3-4.3.21-10a4fa6/Linux64/psfex/16.0+4/lib:/software/lsstsw/stack3_20171023/stack/miniconda3-4.3.21-10a4fa6/Linux64/meas_extensions_psfex/16.0-8-gc315727+6/lib:/software/lsstsw/stack3_20171023/stack/miniconda3-4.3.21-10a4fa6/Linux64/xpa/2.1.15.lsst3/lib:/software/lsstsw/stack3_20171023/stack/miniconda3-4.3.21-10a4fa6/Linux64/meas_extensions_simpleShape/16.0-5-gb3f8a4b+16/lib:/software/lsstsw/stack3_20171023/stack/miniconda3-4.3.21-10a4fa6/Linux64/coadd_chisquared/16.0-2-g839ba83+22/lib:/software/lsstsw/stack3_20171023/stack/miniconda3-4.3.21-10a4fa6/Linux64/ip_diffim/16.0-8-g4aca173+17/lib:/software/lsstsw/stack3_20171023/stack/miniconda3-4.3.21-10a4fa6/Linux64/ip_isr/16.0-7-g2c7e7ec+8/lib:/software/lsstsw/stack3_20171023/stack/miniconda3-4.3.21-10a4fa6/Linux64/meas_extensions_astrometryNet/16.0-4-g347813e+16/lib:/software/lsstsw/stack3_20171023/stack/miniconda3-4.3.21-10a4fa6/Linux64/wcslib/5.13.lsst1+2/lib:/software/lsstsw/stack3_20171023/stack/miniconda3-4.3.21-10a4fa6/Linux64/astrometry_net/0.67.123ff3e.lsst1+8/lib:/software/lsstsw/stack3_20171023/stack/miniconda3-4.3.21-10a4fa6/Linux64/meas_astrom/16.0-10-g7547e25+2/lib:/software/lsstsw/stack3_20171023/stack/miniconda3-4.3.21-10a4fa6/Linux64/shapelet/16.0-6-gf0acd13+8/lib:/software/lsstsw/stack3_20171023/stack/miniconda3-4.3.21-10a4fa6/Linux64/meas_modelfit/16.0-11-gce733cf+7/lib:/software/lsstsw/stack3_20171023/stack/miniconda3-4.3.21-10a4fa6/Linux64/coadd_utils/16.0-4-g8a0f11a+6/lib:/software/lsstsw/stack3_20171023/stack/miniconda3-4.3.21-10a4fa6/Linux64/meas_base/16.0-11-gf9130ea+15/lib:/software/lsstsw/stack3_20171023/stack/miniconda3-4.3.21-10a4fa6/Linux64/meas_algorithms/16.0-13-g0fce7516+2/lib:/software/lsstsw/stack3_20171023/stack/miniconda3-4.3.21-10a4fa6/Linux64/starlink_ast/lsst-dev-gd6cc4e835a/lib:/software/lsstsw/stack3_20171023/stack/miniconda3-4.3.21-10a4fa6/Linux64/astshim/16.0-2-g0febb12+5/lib:/software/lsstsw/stack3_20171023/stack/miniconda3-4.3.21-10a4fa6/Linux64/gsl/2.4/lib:/software/lsstsw/stack3_20171023/stack/miniconda3-4.3.21-10a4fa6/Linux64/minuit2/5.34.14/lib:/software/lsstsw/stack3_20171023/stack/miniconda3-4.3.21-10a4fa6/Linux64/cfitsio/3360.lsst5/lib:/software/lsstsw/stack3_20171023/stack/miniconda3-4.3.21-10a4fa6/Linux64/sphgeom/16.0-4-g5f3a788+5/lib:/software/lsstsw/stack3_20171023/stack/miniconda3-4.3.21-10a4fa6/Linux64/fftw/3.3.4.lsst2/lib:/software/lsstsw/stack3_20171023/stack/miniconda3-4.3.21-10a4fa6/Linux64/geom/16.0-8-g80699e5+1/lib:/software/lsstsw/stack3_20171023/stack/miniconda3-4.3.21-10a4fa6/Linux64/pex_config/16.0-3-g9645794+3/lib:/software/lsstsw/stack3_20171023/stack/miniconda3-4.3.21-10a4fa6/Linux64/libyaml/0.1.7/lib:/software/lsstsw/stack3_20171023/stack/miniconda3-4.3.21-10a4fa6/Linux64/pex_policy/16.0-2-gf41ba6b+3/lib:/software/lsstsw/stack3_20171023/stack/miniconda3-4.3.21-10a4fa6/Linux64/apr_util/1.5.4/lib:/software/lsstsw/stack3_20171023/stack/miniconda3-4.3.21-10a4fa6/Linux64/apr/1.5.2/lib:/software/lsstsw/stack3_20171023/stack/miniconda3-4.3.21-10a4fa6/Linux64/log4cxx/0.10.0.lsst7/lib:/software/lsstsw/stack3_20171023/stack/miniconda3-4.3.21-10a4fa6/Linux64/log/16.0-1-gce273f5+6/lib:/software/lsstsw/stack3_20171023/stack/miniconda3-4.3.21-10a4fa6/Linux64/daf_persistence/16.0-3-g3806c63+6/lib:/software/lsstsw/stack3_20171023/stack/miniconda3-4.3.21-10a4fa6/Linux64/boost/1.68/lib:/software/lsstsw/stack3_20171023/stack/miniconda3-4.3.21-10a4fa6/Linux64/pex_exceptions/16.0-2-gf4e7cdd+3/lib:/software/lsstsw/stack3_20171023/stack/miniconda3-4.3.21-10a4fa6/Linux64/base/16.0-3-g8e51203/lib:/software/lsstsw/stack3_20171023/stack/miniconda3-4.3.21-10a4fa6/Linux64/utils/16.0-6-g3610b4f+3/lib:/software/lsstsw/stack3_20171023/stack/miniconda3-4.3.21-10a4fa6/Linux64/daf_base/16.0-4-g50d071e+6/lib:/software/lsstsw/stack3_20171023/stack/miniconda3-4.3.21-10a4fa6/Linux64/afw/16.0-31-gd4f695684/lib:/software/lsstsw/stack3_20171023/stack/miniconda3-4.3.21-10a4fa6/Linux64/meas_deblender/16.0-7-g37292d5+6/lib
LSST_OBS_DIR=/software/lsstsw/stack3_20171023/stack/miniconda3-4.3.21-10a4fa6/Linux64/lsst_obs/15.0+99
LS_COLORS=rs=0:di=01;34:ln=01;36:mh=00:pi=40;33:so=01;35:do=01;35:bd=40;33;01:cd=40;33;01:or=40;31;01:mi=01;05;37;41:su=37;41:sg=30;43:ca=30;41:tw=30;42:ow=34;42:st=37;44:ex=01;32:*.tar=01;31:*.tgz=01;31:*.arc=01;31:*.arj=01;31:*.taz=01;31:*.lha=01;31:*.lz4=01;31:*.lzh=01;31:*.lzma=01;31:*.tlz=01;31:*.txz=01;31:*.tzo=01;31:*.t7z=01;31:*.zip=01;31:*.z=01;31:*.Z=01;31:*.dz=01;31:*.gz=01;31:*.lrz=01;31:*.lz=01;31:*.lzo=01;31:*.xz=01;31:*.bz2=01;31:*.bz=01;31:*.tbz=01;31:*.tbz2=01;31:*.tz=01;31:*.deb=01;31:*.rpm=01;31:*.jar=01;31:*.war=01;31:*.ear=01;31:*.sar=01;31:*.rar=01;31:*.alz=01;31:*.ace=01;31:*.zoo=01;31:*.cpio=01;31:*.7z=01;31:*.rz=01;31:*.cab=01;31:*.jpg=01;35:*.jpeg=01;35:*.gif=01;35:*.bmp=01;35:*.pbm=01;35:*.pgm=01;35:*.ppm=01;35:*.tga=01;35:*.xbm=01;35:*.xpm=01;35:*.tif=01;35:*.tiff=01;35:*.png=01;35:*.svg=01;35:*.svgz=01;35:*.mng=01;35:*.pcx=01;35:*.mov=01;35:*.mpg=01;35:*.mpeg=01;35:*.m2v=01;35:*.mkv=01;35:*.webm=01;35:*.ogm=01;35:*.mp4=01;35:*.m4v=01;35:*.mp4v=01;35:*.vob=01;35:*.qt=01;35:*.nuv=01;35:*.wmv=01;35:*.asf=01;35:*.rm=01;35:*.rmvb=01;35:*.flc=01;35:*.avi=01;35:*.fli=01;35:*.flv=01;35:*.gl=01;35:*.dl=01;35:*.xcf=01;35:*.xwd=01;35:*.yuv=01;35:*.cgm=01;35:*.emf=01;35:*.axv=01;35:*.anx=01;35:*.ogv=01;35:*.ogx=01;35:*.aac=01;36:*.au=01;36:*.flac=01;36:*.mid=01;36:*.midi=01;36:*.mka=01;36:*.mp3=01;36:*.mpc=01;36:*.ogg=01;36:*.ra=01;36:*.wav=01;36:*.axa=01;36:*.oga=01;36:*.spx=01;36:*.xspf=01;36:
MANPATH=/opt/rh/devtoolset-6/root/usr/share/man::/opt/puppetlabs/puppet/share/man
MATPLOTLIB_DIR=/software/lsstsw/stack3_20171023/stack/miniconda3-4.3.21-10a4fa6/Linux64/matplotlib/2.0.2+2
MEAS_ALGORITHMS_DIR=/software/lsstsw/stack3_20171023/stack/miniconda3-4.3.21-10a4fa6/Linux64/meas_algorithms/16.0-13-g0fce7516+2
MEAS_ASTROM_DIR=/software/lsstsw/stack3_20171023/stack/miniconda3-4.3.21-10a4fa6/Linux64/meas_astrom/16.0-10-g7547e25+2
MEAS_BASE_DIR=/software/lsstsw/stack3_20171023/stack/miniconda3-4.3.21-10a4fa6/Linux64/meas_base/16.0-11-gf9130ea+15
MEAS_DEBLENDER_DIR=/software/lsstsw/stack3_20171023/stack/miniconda3-4.3.21-10a4fa6/Linux64/meas_deblender/16.0-7-g37292d5+6
MEAS_EXTENSIONS_ASTROMETRYNET_DIR=/software/lsstsw/stack3_20171023/stack/miniconda3-4.3.21-10a4fa6/Linux64/meas_extensions_astrometryNet/16.0-4-g347813e+16
MEAS_EXTENSIONS_CONVOLVED_DIR=/software/lsstsw/stack3_20171023/stack/miniconda3-4.3.21-10a4fa6/Linux64/meas_extensions_convolved/16.0-3-ga36a39f+6
MEAS_EXTENSIONS_PHOTOMETRYKRON_DIR=/software/lsstsw/stack3_20171023/stack/miniconda3-4.3.21-10a4fa6/Linux64/meas_extensions_photometryKron/16.0-4-ga5d8928+6
MEAS_EXTENSIONS_PSFEX_DIR=/software/lsstsw/stack3_20171023/stack/miniconda3-4.3.21-10a4fa6/Linux64/meas_extensions_psfex/16.0-8-gc315727+6
MEAS_EXTENSIONS_SHAPEHSM_DIR=/software/lsstsw/stack3_20171023/stack/miniconda3-4.3.21-10a4fa6/Linux64/meas_extensions_shapeHSM/16.0-5-g3a22acd+6
MEAS_EXTENSIONS_SIMPLESHAPE_DIR=/software/lsstsw/stack3_20171023/stack/miniconda3-4.3.21-10a4fa6/Linux64/meas_extensions_simpleShape/16.0-5-gb3f8a4b+16
MEAS_MODELFIT_DIR=/software/lsstsw/stack3_20171023/stack/miniconda3-4.3.21-10a4fa6/Linux64/meas_modelfit/16.0-11-gce733cf+7
MEAS_MOSAIC_DIR=/software/lsstsw/stack3_20171023/stack/miniconda3-4.3.21-10a4fa6/Linux64/meas_mosaic/16.0-10-g3775aa7+2
MINUIT2_DIR=/software/lsstsw/stack3_20171023/stack/miniconda3-4.3.21-10a4fa6/Linux64/minuit2/5.34.14
MPI4PY_DIR=/software/lsstsw/stack3_20171023/stack/miniconda3-4.3.21-10a4fa6/Linux64/mpi4py/2.0.0+6
MPICH_DIR=/software/lsstsw/stack3_20171023/stack/miniconda3-4.3.21-10a4fa6/Linux64/mpich/3.2.1
MPI_DIR=/software/lsstsw/stack3_20171023/stack/miniconda3-4.3.21-10a4fa6/Linux64/mpi/0.0.1+3
NDARRAY_DIR=/software/lsstsw/stack3_20171023/stack/miniconda3-4.3.21-10a4fa6/Linux64/ndarray/1.5.1.lsst1+5
NUMPY_DIR=/software/lsstsw/stack3_20171023/stack/miniconda3-4.3.21-10a4fa6/Linux64/numpy/1.13.1+2
OBS_BASE_DIR=/software/lsstsw/stack3_20171023/stack/miniconda3-4.3.21-10a4fa6/Linux64/obs_base/16.0-11-gfbb8ea7
OBS_CFHT_DIR=/software/lsstsw/stack3_20171023/stack/miniconda3-4.3.21-10a4fa6/Linux64/obs_cfht/16.0-2-gd82908a+30
OBS_COMCAM_DIR=/software/lsstsw/stack3_20171023/stack/miniconda3-4.3.21-10a4fa6/Linux64/obs_comCam/16.0-2-gea03357+1
OBS_CTIO0M9_DIR=/software/lsstsw/stack3_20171023/stack/miniconda3-4.3.21-10a4fa6/Linux64/obs_ctio0m9/16.0-1-g3247260+33
OBS_DECAM_DIR=/software/lsstsw/stack3_20171023/stack/miniconda3-4.3.21-10a4fa6/Linux64/obs_decam/16.0-4-gb463f2d+17
OBS_LSSTSIM_DIR=/software/lsstsw/stack3_20171023/stack/miniconda3-4.3.21-10a4fa6/Linux64/obs_lsstSim/16.0-4-g7986501+19
OBS_MONOCAM_DIR=/software/lsstsw/stack3_20171023/stack/miniconda3-4.3.21-10a4fa6/Linux64/obs_monocam/16.0-2-g379f3ce+11
OBS_SDSS_DIR=/software/lsstsw/stack3_20171023/stack/miniconda3-4.3.21-10a4fa6/Linux64/obs_sdss/16.0-3-g44ea68f+31
OBS_SUBARU_DIR=/software/lsstsw/stack3_20171023/stack/miniconda3-4.3.21-10a4fa6/Linux64/obs_subaru/16.0-20-gb58d072b
OBS_TEST_DIR=/software/lsstsw/stack3_20171023/stack/miniconda3-4.3.21-10a4fa6/Linux64/obs_test/16.0-4-ge3254b7+8
OMP_NUM_THREADS=1
PATH=/software/lsstsw/stack3_20171023/stack/miniconda3-4.3.21-10a4fa6/Linux64/meas_mosaic/16.0-10-g3775aa7+2/bin:/software/lsstsw/stack3_20171023/stack/miniconda3-4.3.21-10a4fa6/Linux64/treecorr/3.2.3.lsst3+3/bin:/software/lsstsw/stack3_20171023/stack/miniconda3-4.3.21-10a4fa6/Linux64/validate_drp/16.0-9-gf51232b+6/bin:/software/lsstsw/stack3_20171023/stack/miniconda3-4.3.21-10a4fa6/Linux64/verify/16.0-1-g0ee5b77+6/bin:/software/lsstsw/stack3_20171023/stack/miniconda3-4.3.21-10a4fa6/Linux64/synpipe/16.0-4-g7690030+17/bin:/software/lsstsw/stack3_20171023/stack/miniconda3-4.3.21-10a4fa6/Linux64/obs_monocam/16.0-2-g379f3ce+11/bin:/software/lsstsw/stack3_20171023/stack/miniconda3-4.3.21-10a4fa6/Linux64/obs_subaru/16.0-20-gb58d072b/bin:/software/lsstsw/stack3_20171023/stack/miniconda3-4.3.21-10a4fa6/Linux64/obs_decam/16.0-4-gb463f2d+17/bin:/software/lsstsw/stack3_20171023/stack/miniconda3-4.3.21-10a4fa6/Linux64/obs_ctio0m9/16.0-1-g3247260+33/bin:/software/lsstsw/stack3_20171023/stack/miniconda3-4.3.21-10a4fa6/Linux64/obs_comCam/16.0-2-gea03357+1/bin:/software/lsstsw/stack3_20171023/stack/miniconda3-4.3.21-10a4fa6/Linux64/obs_cfht/16.0-2-gd82908a+30/bin:/software/lsstsw/stack3_20171023/stack/miniconda3-4.3.21-10a4fa6/Linux64/mpich/3.2.1/bin:/software/lsstsw/stack3_20171023/stack/miniconda3-4.3.21-10a4fa6/Linux64/ctrl_pool/16.0-1-g14407aa+17/bin:/software/lsstsw/stack3_20171023/stack/miniconda3-4.3.21-10a4fa6/Linux64/pipe_drivers/16.0-4-g17aeff3/bin:/software/lsstsw/stack3_20171023/stack/miniconda3-4.3.21-10a4fa6/Linux64/tmv/0.73.lsst2+2/bin:/software/lsstsw/stack3_20171023/stack/miniconda3-4.3.21-10a4fa6/Linux64/galsim/1.6.0.lsst2-1-gd5e7736+1/bin:/software/lsstsw/stack3_20171023/stack/miniconda3-4.3.21-10a4fa6/Linux64/ctrl_orca/16.0-1-g9633d82+9/bin:/software/lsstsw/stack3_20171023/stack/miniconda3-4.3.21-10a4fa6/Linux64/ctrl_execute/15.0+34/bin:/software/lsstsw/stack3_20171023/stack/miniconda3-4.3.21-10a4fa6/Linux64/psfex/16.0+4/bin:/software/lsstsw/stack3_20171023/stack/miniconda3-4.3.21-10a4fa6/Linux64/meas_extensions_psfex/16.0-8-gc315727+6/bin:/software/lsstsw/stack3_20171023/stack/miniconda3-4.3.21-10a4fa6/Linux64/xpa/2.1.15.lsst3/bin:/software/lsstsw/stack3_20171023/stack/miniconda3-4.3.21-10a4fa6/Linux64/obs_sdss/16.0-3-g44ea68f+31/bin:/software/lsstsw/stack3_20171023/stack/miniconda3-4.3.21-10a4fa6/Linux64/obs_lsstSim/16.0-4-g7986501+19/bin:/software/lsstsw/stack3_20171023/stack/miniconda3-4.3.21-10a4fa6/Linux64/pykg_config/1.3.0+4/bin:/software/lsstsw/stack3_20171023/stack/miniconda3-4.3.21-10a4fa6/Linux64/ip_isr/16.0-7-g2c7e7ec+8/bin:/software/lsstsw/stack3_20171023/stack/miniconda3-4.3.21-10a4fa6/Linux64/python_future/0.16.0+3/bin:/software/lsstsw/stack3_20171023/stack/miniconda3-4.3.21-10a4fa6/Linux64/meas_extensions_astrometryNet/16.0-4-g347813e+16/bin:/software/lsstsw/stack3_20171023/stack/miniconda3-4.3.21-10a4fa6/Linux64/astrometry_net/0.67.123ff3e.lsst1+8/bin:/software/lsstsw/stack3_20171023/stack/miniconda3-4.3.21-10a4fa6/Linux64/meas_astrom/16.0-10-g7547e25+2/bin:/software/lsstsw/stack3_20171023/stack/miniconda3-4.3.21-10a4fa6/Linux64/pipe_tasks/16.0-12-g85846a17/bin:/software/lsstsw/stack3_20171023/stack/miniconda3-4.3.21-10a4fa6/Linux64/meas_modelfit/16.0-11-gce733cf+7/bin:/software/lsstsw/stack3_20171023/stack/miniconda3-4.3.21-10a4fa6/Linux64/daf_butler/master-gde6ace3019/bin:/software/lsstsw/stack3_20171023/stack/miniconda3-4.3.21-10a4fa6/Linux64/meas_base/16.0-11-gf9130ea+15/bin:/software/lsstsw/stack3_20171023/stack/miniconda3-4.3.21-10a4fa6/Linux64/starlink_ast/lsst-dev-gd6cc4e835a/bin:/software/lsstsw/stack3_20171023/stack/miniconda3-4.3.21-10a4fa6/Linux64/gsl/2.4/bin:/software/lsstsw/stack3_20171023/stack/miniconda3-4.3.21-10a4fa6/Linux64/cfitsio/3360.lsst5/bin:/software/lsstsw/stack3_20171023/stack/miniconda3-4.3.21-10a4fa6/Linux64/fftw/3.3.4.lsst2/bin:/software/lsstsw/stack3_20171023/stack/miniconda3-4.3.21-10a4fa6/Linux64/pex_config/16.0-3-g9645794+3/bin:/software/lsstsw/stack3_20171023/stack/miniconda3-4.3.21-10a4fa6/Linux64/pyyaml/3.13/bin:/software/lsstsw/stack3_20171023/stack/miniconda3-4.3.21-10a4fa6/Linux64/pex_policy/16.0-2-gf41ba6b+3/bin:/software/lsstsw/stack3_20171023/stack/miniconda3-4.3.21-10a4fa6/Linux64/apr_util/1.5.4/bin:/software/lsstsw/stack3_20171023/stack/miniconda3-4.3.21-10a4fa6/Linux64/apr/1.5.2/bin:/software/lsstsw/stack3_20171023/stack/miniconda3-4.3.21-10a4fa6/Linux64/python_psutil/5.4.3+2/bin:/software/lsstsw/stack3_20171023/stack/miniconda3-4.3.21-10a4fa6/Linux64/boost/1.68/bin:/software/lsstsw/stack3_20171023/stack/miniconda3-4.3.21-10a4fa6/Linux64/doxygen/1.8.13.lsst2/bin:/software/lsstsw/stack3_20171023/stack/miniconda3-4.3.21-10a4fa6/Linux64/python_coverage/4.5.1/bin:/software/lsstsw/stack3_20171023/stack/miniconda3-4.3.21-10a4fa6/Linux64/pep8_naming/0.4.1+3/bin:/software/lsstsw/stack3_20171023/stack/miniconda3-4.3.21-10a4fa6/Linux64/pyflakes/1.6.0+2/bin:/software/lsstsw/stack3_20171023/stack/miniconda3-4.3.21-10a4fa6/Linux64/flake8/3.5.0+8/bin:/software/lsstsw/stack3_20171023/stack/miniconda3-4.3.21-10a4fa6/Linux64/pytest/3.2.0.lsst4+1/bin:/software/lsstsw/stack3_20171023/stack/miniconda3-4.3.21-10a4fa6/Linux64/scons/3.0.0.lsst1+3/bin:/software/lsstsw/stack3_20171023/stack/miniconda3-4.3.21-10a4fa6/Linux64/sconsUtils/16.0-5-ga6f992e/bin:/software/lsstsw/stack3_20171023/stack/miniconda3-4.3.21-10a4fa6/Linux64/afw/16.0-31-gd4f695684/bin:/software/lsstsw/stack/eups/2.1.4/bin:/software/lsstsw/stack3_20171023/python/miniconda3-4.3.21/bin:/opt/rh/devtoolset-6/root/usr/bin:/usr/local/bin:/usr/bin:/usr/local/sbin:/usr/sbin:/opt/puppetlabs/bin:/opt/dell/srvadmin/bin:/home/$USER/.local/bin:/home/$USER/bin:/software/lsstsw/stack3_20171023/stack/miniconda3-4.3.21-10a4fa6/Linux64/jointcal/16.0-15-g8e16a51+1/bin
PCP_DIR=/opt/rh/devtoolset-6/root
PEP8_NAMING_DIR=/software/lsstsw/stack3_20171023/stack/miniconda3-4.3.21-10a4fa6/Linux64/pep8_naming/0.4.1+3
PERL5LIB=/opt/rh/devtoolset-6/root//usr/lib64/perl5/vendor_perl:/opt/rh/devtoolset-6/root/usr/lib/perl5:/opt/rh/devtoolset-6/root//usr/share/perl5/vendor_perl
PEX_CONFIG_DIR=/software/lsstsw/stack3_20171023/stack/miniconda3-4.3.21-10a4fa6/Linux64/pex_config/16.0-3-g9645794+3
PEX_EXCEPTIONS_DIR=/software/lsstsw/stack3_20171023/stack/miniconda3-4.3.21-10a4fa6/Linux64/pex_exceptions/16.0-2-gf4e7cdd+3
PEX_POLICY_DIR=/software/lsstsw/stack3_20171023/stack/miniconda3-4.3.21-10a4fa6/Linux64/pex_policy/16.0-2-gf41ba6b+3
PIPE_BASE_DIR=/software/lsstsw/stack3_20171023/stack/miniconda3-4.3.21-10a4fa6/Linux64/pipe_base/16.0-11-g9fe0e56+7
PIPE_DRIVERS_DIR=/software/lsstsw/stack3_20171023/stack/miniconda3-4.3.21-10a4fa6/Linux64/pipe_drivers/16.0-4-g17aeff3
PIPE_TASKS_DIR=/software/lsstsw/stack3_20171023/stack/miniconda3-4.3.21-10a4fa6/Linux64/pipe_tasks/16.0-12-g85846a17
PROXMIN_DIR=/software/lsstsw/stack3_20171023/stack/miniconda3-4.3.21-10a4fa6/Linux64/proxmin/lsst-dev-g87d7c0f1bb
PSFEX_DIR=/software/lsstsw/stack3_20171023/stack/miniconda3-4.3.21-10a4fa6/Linux64/psfex/16.0+4
PYBIND11_DIR=/software/lsstsw/stack3_20171023/stack/miniconda3-4.3.21-10a4fa6/Linux64/pybind11/2.2.3.lsst1
PYCODESTYLE_DIR=/software/lsstsw/stack3_20171023/stack/miniconda3-4.3.21-10a4fa6/Linux64/pycodestyle/2.3.1+3
PYFLAKES_DIR=/software/lsstsw/stack3_20171023/stack/miniconda3-4.3.21-10a4fa6/Linux64/pyflakes/1.6.0+2
PYKG_CONFIG_DIR=/software/lsstsw/stack3_20171023/stack/miniconda3-4.3.21-10a4fa6/Linux64/pykg_config/1.3.0+4
PYTEST_COV_DIR=/software/lsstsw/stack3_20171023/stack/miniconda3-4.3.21-10a4fa6/Linux64/pytest_cov/2.5.1
PYTEST_DIR=/software/lsstsw/stack3_20171023/stack/miniconda3-4.3.21-10a4fa6/Linux64/pytest/3.2.0.lsst4+1
PYTEST_FLAKE8_DIR=/software/lsstsw/stack3_20171023/stack/miniconda3-4.3.21-10a4fa6/Linux64/pytest_flake8/0.9.1+8
PYTEST_FORKED_DIR=/software/lsstsw/stack3_20171023/stack/miniconda3-4.3.21-10a4fa6/Linux64/pytest_forked/0.2.lsst4+1
PYTEST_SESSION2FILE_DIR=/software/lsstsw/stack3_20171023/stack/miniconda3-4.3.21-10a4fa6/Linux64/pytest_session2file/0.1.9+9
PYTEST_XDIST_DIR=/software/lsstsw/stack3_20171023/stack/miniconda3-4.3.21-10a4fa6/Linux64/pytest_xdist/1.20.1.lsst4+1
PYTHONPATH=/software/lsstsw/stack3_20171023/stack/miniconda3-4.3.21-10a4fa6/Linux64/meas_mosaic/16.0-10-g3775aa7+2/python:/software/lsstsw/stack3_20171023/stack/miniconda3-4.3.21-10a4fa6/Linux64/treecorr/3.2.3.lsst3+3/lib/python:/software/lsstsw/stack3_20171023/stack/miniconda3-4.3.21-10a4fa6/Linux64/validate_drp/16.0-9-gf51232b+6/python:/software/lsstsw/stack3_20171023/stack/miniconda3-4.3.21-10a4fa6/Linux64/cp_pipe/16.0+32/python:/software/lsstsw/stack3_20171023/stack/miniconda3-4.3.21-10a4fa6/Linux64/display_matplotlib/16.0-3-gcfd6c53+14/python:/software/lsstsw/stack3_20171023/stack/miniconda3-4.3.21-10a4fa6/Linux64/ws4py/0.4.2+3/lib/python:/software/lsstsw/stack3_20171023/stack/miniconda3-4.3.21-10a4fa6/Linux64/firefly_client/lsst-dev-gd3d76961fa/lib/python:/software/lsstsw/stack3_20171023/stack/miniconda3-4.3.21-10a4fa6/Linux64/display_firefly/16.0-4-g03cf288+5/python:/software/lsstsw/stack3_20171023/stack/miniconda3-4.3.21-10a4fa6/Linux64/verify_metrics/16.0-2-g16e3d1c+3/python:/software/lsstsw/stack3_20171023/stack/miniconda3-4.3.21-10a4fa6/Linux64/requests/2.9.1.lsst1+4/lib/python:/software/lsstsw/stack3_20171023/stack/miniconda3-4.3.21-10a4fa6/Linux64/verify/16.0-1-g0ee5b77+6/python:/software/lsstsw/stack3_20171023/stack/miniconda3-4.3.21-10a4fa6/Linux64/jointcal/16.0-15-g8e16a51+1/python:/software/lsstsw/stack3_20171023/stack/miniconda3-4.3.21-10a4fa6/Linux64/synpipe/16.0-4-g7690030+17/python:/software/lsstsw/stack3_20171023/stack/miniconda3-4.3.21-10a4fa6/Linux64/obs_monocam/16.0-2-g379f3ce+11/python:/software/lsstsw/stack3_20171023/stack/miniconda3-4.3.21-10a4fa6/Linux64/obs_subaru/16.0-20-gb58d072b/python:/software/lsstsw/stack3_20171023/stack/miniconda3-4.3.21-10a4fa6/Linux64/obs_decam/16.0-4-gb463f2d+17/python:/software/lsstsw/stack3_20171023/stack/miniconda3-4.3.21-10a4fa6/Linux64/obs_ctio0m9/16.0-1-g3247260+33/python:/software/lsstsw/stack3_20171023/stack/miniconda3-4.3.21-10a4fa6/Linux64/obs_comCam/16.0-2-gea03357+1/python:/software/lsstsw/stack3_20171023/stack/miniconda3-4.3.21-10a4fa6/Linux64/obs_cfht/16.0-2-gd82908a+30/python:/software/lsstsw/stack3_20171023/stack/miniconda3-4.3.21-10a4fa6/Linux64/mpi4py/2.0.0+6/lib/python:/software/lsstsw/stack3_20171023/stack/miniconda3-4.3.21-10a4fa6/Linux64/ctrl_pool/16.0-1-g14407aa+17/python:/software/lsstsw/stack3_20171023/stack/miniconda3-4.3.21-10a4fa6/Linux64/pipe_drivers/16.0-4-g17aeff3/python:/software/lsstsw/stack3_20171023/stack/miniconda3-4.3.21-10a4fa6/Linux64/galsim/1.6.0.lsst2-1-gd5e7736+1/lib/python:/software/lsstsw/stack3_20171023/stack/miniconda3-4.3.21-10a4fa6/Linux64/galsim/1.6.0.lsst2-1-gd5e7736+1:/software/lsstsw/stack3_20171023/stack/miniconda3-4.3.21-10a4fa6/Linux64/meas_extensions_shapeHSM/16.0-5-g3a22acd+6/python:/software/lsstsw/stack3_20171023/stack/miniconda3-4.3.21-10a4fa6/Linux64/meas_extensions_photometryKron/16.0-4-ga5d8928+6/python:/software/lsstsw/stack3_20171023/stack/miniconda3-4.3.21-10a4fa6/Linux64/ctrl_orca/16.0-1-g9633d82+9/python:/software/lsstsw/stack3_20171023/stack/miniconda3-4.3.21-10a4fa6/Linux64/ctrl_execute/15.0+34/python:/software/lsstsw/stack3_20171023/stack/miniconda3-4.3.21-10a4fa6/Linux64/psfex/16.0+4/python:/software/lsstsw/stack3_20171023/stack/miniconda3-4.3.21-10a4fa6/Linux64/meas_extensions_psfex/16.0-8-gc315727+6/python:/software/lsstsw/stack3_20171023/stack/miniconda3-4.3.21-10a4fa6/Linux64/display_ds9/16.0-2-g9d5294e+16/python:/software/lsstsw/stack3_20171023/stack/miniconda3-4.3.21-10a4fa6/Linux64/meas_extensions_simpleShape/16.0-5-gb3f8a4b+16/python:/software/lsstsw/stack3_20171023/stack/miniconda3-4.3.21-10a4fa6/Linux64/obs_sdss/16.0-3-g44ea68f+31/python:/software/lsstsw/stack3_20171023/stack/miniconda3-4.3.21-10a4fa6/Linux64/obs_lsstSim/16.0-4-g7986501+19/python:/software/lsstsw/stack3_20171023/stack/miniconda3-4.3.21-10a4fa6/Linux64/pykg_config/1.3.0+4/lib/python:/software/lsstsw/stack3_20171023/stack/miniconda3-4.3.21-10a4fa6/Linux64/healpy/1.10.3.lsst1+6/lib/python:/software/lsstsw/stack3_20171023/stack/miniconda3-4.3.21-10a4fa6/Linux64/skymap/16.0-3-g6923fb6+9/python:/software/lsstsw/stack3_20171023/stack/miniconda3-4.3.21-10a4fa6/Linux64/coadd_chisquared/16.0-2-g839ba83+22/python:/software/lsstsw/stack3_20171023/stack/miniconda3-4.3.21-10a4fa6/Linux64/lmfit/0.9.3+8/lib/python:/software/lsstsw/stack3_20171023/stack/miniconda3-4.3.21-10a4fa6/Linux64/ip_diffim/16.0-8-g4aca173+17/python:/software/lsstsw/stack3_20171023/stack/miniconda3-4.3.21-10a4fa6/Linux64/ip_isr/16.0-7-g2c7e7ec+8/python:/software/lsstsw/stack3_20171023/stack/miniconda3-4.3.21-10a4fa6/Linux64/python_future/0.16.0+3/lib/python:/software/lsstsw/stack3_20171023/stack/miniconda3-4.3.21-10a4fa6/Linux64/meas_extensions_astrometryNet/16.0-4-g347813e+16/python:/software/lsstsw/stack3_20171023/stack/miniconda3-4.3.21-10a4fa6/Linux64/astrometry_net/0.67.123ff3e.lsst1+8/lib/python:/software/lsstsw/stack3_20171023/stack/miniconda3-4.3.21-10a4fa6/Linux64/meas_astrom/16.0-10-g7547e25+2/python:/software/lsstsw/stack3_20171023/stack/miniconda3-4.3.21-10a4fa6/Linux64/pipe_tasks/16.0-12-g85846a17/python:/software/lsstsw/stack3_20171023/stack/miniconda3-4.3.21-10a4fa6/Linux64/shapelet/16.0-6-gf0acd13+8/python:/software/lsstsw/stack3_20171023/stack/miniconda3-4.3.21-10a4fa6/Linux64/meas_modelfit/16.0-11-gce733cf+7/python:/software/lsstsw/stack3_20171023/stack/miniconda3-4.3.21-10a4fa6/Linux64/proxmin/lsst-dev-g87d7c0f1bb/lib/python:/software/lsstsw/stack3_20171023/stack/miniconda3-4.3.21-10a4fa6/Linux64/scarlet/lsst-dev-gfdfbe1c240+1/lib/python:/software/lsstsw/stack3_20171023/stack/miniconda3-4.3.21-10a4fa6/Linux64/obs_base/16.0-11-gfbb8ea7/python:/software/lsstsw/stack3_20171023/stack/miniconda3-4.3.21-10a4fa6/Linux64/obs_test/16.0-4-ge3254b7+8/python:/software/lsstsw/stack3_20171023/stack/miniconda3-4.3.21-10a4fa6/Linux64/sqlalchemy/1.2.2+2/lib/python:/software/lsstsw/stack3_20171023/stack/miniconda3-4.3.21-10a4fa6/Linux64/daf_butler/master-gde6ace3019/python:/software/lsstsw/stack3_20171023/stack/miniconda3-4.3.21-10a4fa6/Linux64/pipe_base/16.0-11-g9fe0e56+7/python:/software/lsstsw/stack3_20171023/stack/miniconda3-4.3.21-10a4fa6/Linux64/coadd_utils/16.0-4-g8a0f11a+6/python:/software/lsstsw/stack3_20171023/stack/miniconda3-4.3.21-10a4fa6/Linux64/meas_base/16.0-11-gf9130ea+15/python:/software/lsstsw/stack3_20171023/stack/miniconda3-4.3.21-10a4fa6/Linux64/esutil/0.6.2.5.lsst1+2/lib/python:/software/lsstsw/stack3_20171023/stack/miniconda3-4.3.21-10a4fa6/Linux64/meas_algorithms/16.0-13-g0fce7516+2/python:/software/lsstsw/stack3_20171023/stack/miniconda3-4.3.21-10a4fa6/Linux64/astshim/16.0-2-g0febb12+5/python:/software/lsstsw/stack3_20171023/stack/miniconda3-4.3.21-10a4fa6/Linux64/sphgeom/16.0-4-g5f3a788+5/python:/software/lsstsw/stack3_20171023/stack/miniconda3-4.3.21-10a4fa6/Linux64/ndarray/1.5.1.lsst1+5/python:/software/lsstsw/stack3_20171023/stack/miniconda3-4.3.21-10a4fa6/Linux64/geom/16.0-8-g80699e5+1/python:/software/lsstsw/stack3_20171023/stack/miniconda3-4.3.21-10a4fa6/Linux64/pex_config/16.0-3-g9645794+3/python:/software/lsstsw/stack3_20171023/stack/miniconda3-4.3.21-10a4fa6/Linux64/pyyaml/3.13/lib/python:/software/lsstsw/stack3_20171023/stack/miniconda3-4.3.21-10a4fa6/Linux64/pex_policy/16.0-2-gf41ba6b+3/python:/software/lsstsw/stack3_20171023/stack/miniconda3-4.3.21-10a4fa6/Linux64/log/16.0-1-gce273f5+6/python:/software/lsstsw/stack3_20171023/stack/miniconda3-4.3.21-10a4fa6/Linux64/daf_persistence/16.0-3-g3806c63+6/python:/software/lsstsw/stack3_20171023/stack/miniconda3-4.3.21-10a4fa6/Linux64/python_psutil/5.4.3+2/lib/python:/software/lsstsw/stack3_20171023/stack/miniconda3-4.3.21-10a4fa6/Linux64/pex_exceptions/16.0-2-gf4e7cdd+3/python:/software/lsstsw/stack3_20171023/stack/miniconda3-4.3.21-10a4fa6/Linux64/python_coverage/4.5.1/lib/python:/software/lsstsw/stack3_20171023/stack/miniconda3-4.3.21-10a4fa6/Linux64/pytest_cov/2.5.1/lib/python:/software/lsstsw/stack3_20171023/stack/miniconda3-4.3.21-10a4fa6/Linux64/pytest_session2file/0.1.9+9/lib/python:/software/lsstsw/stack3_20171023/stack/miniconda3-4.3.21-10a4fa6/Linux64/python_execnet/1.4.1.lsst4+1/lib/python:/software/lsstsw/stack3_20171023/stack/miniconda3-4.3.21-10a4fa6/Linux64/pytest_forked/0.2.lsst4+1/lib/python:/software/lsstsw/stack3_20171023/stack/miniconda3-4.3.21-10a4fa6/Linux64/pytest_xdist/1.20.1.lsst4+1/lib/python:/software/lsstsw/stack3_20171023/stack/miniconda3-4.3.21-10a4fa6/Linux64/pep8_naming/0.4.1+3/lib/python:/software/lsstsw/stack3_20171023/stack/miniconda3-4.3.21-10a4fa6/Linux64/pyflakes/1.6.0+2/lib/python:/software/lsstsw/stack3_20171023/stack/miniconda3-4.3.21-10a4fa6/Linux64/pycodestyle/2.3.1+3/lib/python:/software/lsstsw/stack3_20171023/stack/miniconda3-4.3.21-10a4fa6/Linux64/python_mccabe/0.6.1+9/lib/python:/software/lsstsw/stack3_20171023/stack/miniconda3-4.3.21-10a4fa6/Linux64/flake8/3.5.0+8/lib/python:/software/lsstsw/stack3_20171023/stack/miniconda3-4.3.21-10a4fa6/Linux64/pytest_flake8/0.9.1+8/lib/python:/software/lsstsw/stack3_20171023/stack/miniconda3-4.3.21-10a4fa6/Linux64/pytest/3.2.0.lsst4+1/lib/python:/software/lsstsw/stack3_20171023/stack/miniconda3-4.3.21-10a4fa6/Linux64/sconsUtils/16.0-5-ga6f992e/python:/software/lsstsw/stack3_20171023/stack/miniconda3-4.3.21-10a4fa6/Linux64/base/16.0-3-g8e51203/python:/software/lsstsw/stack3_20171023/stack/miniconda3-4.3.21-10a4fa6/Linux64/utils/16.0-6-g3610b4f+3/python:/software/lsstsw/stack3_20171023/stack/miniconda3-4.3.21-10a4fa6/Linux64/daf_base/16.0-4-g50d071e+6/python:/software/lsstsw/stack3_20171023/stack/miniconda3-4.3.21-10a4fa6/Linux64/afw/16.0-31-gd4f695684/python:/software/lsstsw/stack3_20171023/stack/miniconda3-4.3.21-10a4fa6/Linux64/meas_deblender/16.0-7-g37292d5+6/python:/software/lsstsw/stack/eups/2.1.4/python:/opt/rh/devtoolset-6/root/usr/lib64/python2.7/site-packages:/opt/rh/devtoolset-6/root/usr/lib/python2.7/site-packages:/software/lsstsw/stack3_20171023/stack/miniconda3-4.3.21-10a4fa6/Linux64/meas_extensions_convolved/16.0-3-ga36a39f+6/python
PYTHON_COVERAGE_DIR=/software/lsstsw/stack3_20171023/stack/miniconda3-4.3.21-10a4fa6/Linux64/python_coverage/4.5.1
PYTHON_DIR=/software/lsstsw/stack3_20171023/stack/miniconda3-4.3.21-10a4fa6/Linux64/python/0.0.8
PYTHON_EXECNET_DIR=/software/lsstsw/stack3_20171023/stack/miniconda3-4.3.21-10a4fa6/Linux64/python_execnet/1.4.1.lsst4+1
PYTHON_FUTURE_DIR=/software/lsstsw/stack3_20171023/stack/miniconda3-4.3.21-10a4fa6/Linux64/python_future/0.16.0+3
PYTHON_MCCABE_DIR=/software/lsstsw/stack3_20171023/stack/miniconda3-4.3.21-10a4fa6/Linux64/python_mccabe/0.6.1+9
PYTHON_PSUTIL_DIR=/software/lsstsw/stack3_20171023/stack/miniconda3-4.3.21-10a4fa6/Linux64/python_psutil/5.4.3+2
PYYAML_DIR=/software/lsstsw/stack3_20171023/stack/miniconda3-4.3.21-10a4fa6/Linux64/pyyaml/3.13
QT_GRAPHICSSYSTEM_CHECKED=1
REQUESTS_DIR=/software/lsstsw/stack3_20171023/stack/miniconda3-4.3.21-10a4fa6/Linux64/requests/2.9.1.lsst1+4
RUN_ID=DM-15517
RUN_ID_URL=https://jira.lsstcorp.org/browse/DM-15517
SCARLET_DIR=/software/lsstsw/stack3_20171023/stack/miniconda3-4.3.21-10a4fa6/Linux64/scarlet/lsst-dev-gfdfbe1c240+1
SCIPY_DIR=/software/lsstsw/stack3_20171023/stack/miniconda3-4.3.21-10a4fa6/Linux64/scipy/0.0.1.lsst1+7
SCONSUTILS_DIR=/software/lsstsw/stack3_20171023/stack/miniconda3-4.3.21-10a4fa6/Linux64/sconsUtils/16.0-5-ga6f992e
SCONS_DIR=/software/lsstsw/stack3_20171023/stack/miniconda3-4.3.21-10a4fa6/Linux64/scons/3.0.0.lsst1+3
SETUP_AFW=afw 16.0-31-gd4f695684 -f Linux64 -Z /software/lsstsw/stack3_20171023/stack/miniconda3-4.3.21-10a4fa6
SETUP_APR=apr 1.5.2 -f Linux64 -Z /software/lsstsw/stack3_20171023/stack/miniconda3-4.3.21-10a4fa6
SETUP_APR_UTIL=apr_util 1.5.4 -f Linux64 -Z /software/lsstsw/stack3_20171023/stack/miniconda3-4.3.21-10a4fa6
SETUP_ASTROMETRY_NET=astrometry_net 0.67.123ff3e.lsst1+8 -f Linux64 -Z /software/lsstsw/stack3_20171023/stack/miniconda3-4.3.21-10a4fa6
SETUP_ASTROMETRY_NET_DATA=astrometry_net_data 8.0.0.0+40 -f Linux64 -Z /software/lsstsw/stack3_20171023/stack/miniconda3-4.3.21-10a4fa6
SETUP_ASTROPY=astropy 2.0.1+2 -f Linux64 -Z /software/lsstsw/stack3_20171023/stack/miniconda3-4.3.21-10a4fa6
SETUP_ASTSHIM=astshim 16.0-2-g0febb12+5 -f Linux64 -Z /software/lsstsw/stack3_20171023/stack/miniconda3-4.3.21-10a4fa6
SETUP_BASE=base 16.0-3-g8e51203 -f Linux64 -Z /software/lsstsw/stack3_20171023/stack/miniconda3-4.3.21-10a4fa6
SETUP_BOOST=boost 1.68 -f Linux64 -Z /software/lsstsw/stack3_20171023/stack/miniconda3-4.3.21-10a4fa6
SETUP_CFITSIO=cfitsio 3360.lsst5 -f Linux64 -Z /software/lsstsw/stack3_20171023/stack/miniconda3-4.3.21-10a4fa6
SETUP_COADD_CHISQUARED=coadd_chisquared 16.0-2-g839ba83+22 -f Linux64 -Z /software/lsstsw/stack3_20171023/stack/miniconda3-4.3.21-10a4fa6
SETUP_COADD_UTILS=coadd_utils 16.0-4-g8a0f11a+6 -f Linux64 -Z /software/lsstsw/stack3_20171023/stack/miniconda3-4.3.21-10a4fa6
SETUP_CP_PIPE=cp_pipe 16.0+32 -f Linux64 -Z /software/lsstsw/stack3_20171023/stack/miniconda3-4.3.21-10a4fa6
SETUP_CTRL_EXECUTE=ctrl_execute 15.0+34 -f Linux64 -Z /software/lsstsw/stack3_20171023/stack/miniconda3-4.3.21-10a4fa6
SETUP_CTRL_ORCA=ctrl_orca 16.0-1-g9633d82+9 -f Linux64 -Z /software/lsstsw/stack3_20171023/stack/miniconda3-4.3.21-10a4fa6
SETUP_CTRL_PLATFORM_LSSTVC=ctrl_platform_lsstvc 15.0+34 -f Linux64 -Z /software/lsstsw/stack3_20171023/stack/miniconda3-4.3.21-10a4fa6
SETUP_CTRL_POOL=ctrl_pool 16.0-1-g14407aa+17 -f Linux64 -Z /software/lsstsw/stack3_20171023/stack/miniconda3-4.3.21-10a4fa6
SETUP_DAF_BASE=daf_base 16.0-4-g50d071e+6 -f Linux64 -Z /software/lsstsw/stack3_20171023/stack/miniconda3-4.3.21-10a4fa6
SETUP_DAF_BUTLER=daf_butler master-gde6ace3019 -f Linux64 -Z /software/lsstsw/stack3_20171023/stack/miniconda3-4.3.21-10a4fa6
SETUP_DAF_PERSISTENCE=daf_persistence 16.0-3-g3806c63+6 -f Linux64 -Z /software/lsstsw/stack3_20171023/stack/miniconda3-4.3.21-10a4fa6
SETUP_DISPLAY_DS9=display_ds9 16.0-2-g9d5294e+16 -f Linux64 -Z /software/lsstsw/stack3_20171023/stack/miniconda3-4.3.21-10a4fa6
SETUP_DISPLAY_FIREFLY=display_firefly 16.0-4-g03cf288+5 -f Linux64 -Z /software/lsstsw/stack3_20171023/stack/miniconda3-4.3.21-10a4fa6
SETUP_DISPLAY_MATPLOTLIB=display_matplotlib 16.0-3-gcfd6c53+14 -f Linux64 -Z /software/lsstsw/stack3_20171023/stack/miniconda3-4.3.21-10a4fa6
SETUP_DOXYGEN=doxygen 1.8.13.lsst2 -f Linux64 -Z /software/lsstsw/stack3_20171023/stack/miniconda3-4.3.21-10a4fa6
SETUP_EIGEN=eigen 3.3.4.lsst1 -f Linux64 -Z /software/lsstsw/stack3_20171023/stack/miniconda3-4.3.21-10a4fa6
SETUP_ESUTIL=esutil 0.6.2.5.lsst1+2 -f Linux64 -Z /software/lsstsw/stack3_20171023/stack/miniconda3-4.3.21-10a4fa6
SETUP_EUPS=eups LOCAL:/software/lsstsw/stack/eups/2.1.4 -f (none) -Z (none)
SETUP_FFTW=fftw 3.3.4.lsst2 -f Linux64 -Z /software/lsstsw/stack3_20171023/stack/miniconda3-4.3.21-10a4fa6
SETUP_FIREFLY_CLIENT=firefly_client lsst-dev-gd3d76961fa -f Linux64 -Z /software/lsstsw/stack3_20171023/stack/miniconda3-4.3.21-10a4fa6
SETUP_FLAKE8=flake8 3.5.0+8 -f Linux64 -Z /software/lsstsw/stack3_20171023/stack/miniconda3-4.3.21-10a4fa6
SETUP_GALSIM=galsim 1.6.0.lsst2-1-gd5e7736+1 -f Linux64 -Z /software/lsstsw/stack3_20171023/stack/miniconda3-4.3.21-10a4fa6
SETUP_GEOM=geom 16.0-8-g80699e5+1 -f Linux64 -Z /software/lsstsw/stack3_20171023/stack/miniconda3-4.3.21-10a4fa6
SETUP_GSL=gsl 2.4 -f Linux64 -Z /software/lsstsw/stack3_20171023/stack/miniconda3-4.3.21-10a4fa6
SETUP_HEALPY=healpy 1.10.3.lsst1+6 -f Linux64 -Z /software/lsstsw/stack3_20171023/stack/miniconda3-4.3.21-10a4fa6
SETUP_IP_DIFFIM=ip_diffim 16.0-8-g4aca173+17 -f Linux64 -Z /software/lsstsw/stack3_20171023/stack/miniconda3-4.3.21-10a4fa6
SETUP_IP_ISR=ip_isr 16.0-7-g2c7e7ec+8 -f Linux64 -Z /software/lsstsw/stack3_20171023/stack/miniconda3-4.3.21-10a4fa6
SETUP_JOINTCAL=jointcal 16.0-15-g8e16a51+1 -f Linux64 -Z /software/lsstsw/stack3_20171023/stack/miniconda3-4.3.21-10a4fa6
SETUP_JOINTCAL_CHOLMOD=jointcal_cholmod master-g48cb81d145+44 -f Linux64 -Z /software/lsstsw/stack3_20171023/stack/miniconda3-4.3.21-10a4fa6
SETUP_LIBYAML=libyaml 0.1.7 -f Linux64 -Z /software/lsstsw/stack3_20171023/stack/miniconda3-4.3.21-10a4fa6
SETUP_LMFIT=lmfit 0.9.3+8 -f Linux64 -Z /software/lsstsw/stack3_20171023/stack/miniconda3-4.3.21-10a4fa6
SETUP_LOG4CXX=log4cxx 0.10.0.lsst7 -f Linux64 -Z /software/lsstsw/stack3_20171023/stack/miniconda3-4.3.21-10a4fa6
SETUP_LOG=log 16.0-1-gce273f5+6 -f Linux64 -Z /software/lsstsw/stack3_20171023/stack/miniconda3-4.3.21-10a4fa6
SETUP_LSST_APPS=lsst_apps 14.0+178 -f Linux64 -Z /software/lsstsw/stack3_20171023/stack/miniconda3-4.3.21-10a4fa6
SETUP_LSST_DISTRIB=lsst_distrib 16.0+41 -f Linux64 -Z /software/lsstsw/stack3_20171023/stack/miniconda3-4.3.21-10a4fa6
SETUP_LSST_OBS=lsst_obs 15.0+99 -f Linux64 -Z /software/lsstsw/stack3_20171023/stack/miniconda3-4.3.21-10a4fa6
SETUP_MATPLOTLIB=matplotlib 2.0.2+2 -f Linux64 -Z /software/lsstsw/stack3_20171023/stack/miniconda3-4.3.21-10a4fa6
SETUP_MEAS_ALGORITHMS=meas_algorithms 16.0-13-g0fce7516+2 -f Linux64 -Z /software/lsstsw/stack3_20171023/stack/miniconda3-4.3.21-10a4fa6
SETUP_MEAS_ASTROM=meas_astrom 16.0-10-g7547e25+2 -f Linux64 -Z /software/lsstsw/stack3_20171023/stack/miniconda3-4.3.21-10a4fa6
SETUP_MEAS_BASE=meas_base 16.0-11-gf9130ea+15 -f Linux64 -Z /software/lsstsw/stack3_20171023/stack/miniconda3-4.3.21-10a4fa6
SETUP_MEAS_DEBLENDER=meas_deblender 16.0-7-g37292d5+6 -f Linux64 -Z /software/lsstsw/stack3_20171023/stack/miniconda3-4.3.21-10a4fa6
SETUP_MEAS_EXTENSIONS_ASTROMETRYNET=meas_extensions_astrometryNet 16.0-4-g347813e+16 -f Linux64 -Z /software/lsstsw/stack3_20171023/stack/miniconda3-4.3.21-10a4fa6
SETUP_MEAS_EXTENSIONS_CONVOLVED=meas_extensions_convolved 16.0-3-ga36a39f+6 -f Linux64 -Z /software/lsstsw/stack3_20171023/stack/miniconda3-4.3.21-10a4fa6
SETUP_MEAS_EXTENSIONS_PHOTOMETRYKRON=meas_extensions_photometryKron 16.0-4-ga5d8928+6 -f Linux64 -Z /software/lsstsw/stack3_20171023/stack/miniconda3-4.3.21-10a4fa6
SETUP_MEAS_EXTENSIONS_PSFEX=meas_extensions_psfex 16.0-8-gc315727+6 -f Linux64 -Z /software/lsstsw/stack3_20171023/stack/miniconda3-4.3.21-10a4fa6
SETUP_MEAS_EXTENSIONS_SHAPEHSM=meas_extensions_shapeHSM 16.0-5-g3a22acd+6 -f Linux64 -Z /software/lsstsw/stack3_20171023/stack/miniconda3-4.3.21-10a4fa6
SETUP_MEAS_EXTENSIONS_SIMPLESHAPE=meas_extensions_simpleShape 16.0-5-gb3f8a4b+16 -f Linux64 -Z /software/lsstsw/stack3_20171023/stack/miniconda3-4.3.21-10a4fa6
SETUP_MEAS_MODELFIT=meas_modelfit 16.0-11-gce733cf+7 -f Linux64 -Z /software/lsstsw/stack3_20171023/stack/miniconda3-4.3.21-10a4fa6
SETUP_MEAS_MOSAIC=meas_mosaic 16.0-10-g3775aa7+2 -f Linux64 -Z /software/lsstsw/stack3_20171023/stack/miniconda3-4.3.21-10a4fa6
SETUP_MINUIT2=minuit2 5.34.14 -f Linux64 -Z /software/lsstsw/stack3_20171023/stack/miniconda3-4.3.21-10a4fa6
SETUP_MPI4PY=mpi4py 2.0.0+6 -f Linux64 -Z /software/lsstsw/stack3_20171023/stack/miniconda3-4.3.21-10a4fa6
SETUP_MPI=mpi 0.0.1+3 -f Linux64 -Z /software/lsstsw/stack3_20171023/stack/miniconda3-4.3.21-10a4fa6
SETUP_MPICH=mpich 3.2.1 -f Linux64 -Z /software/lsstsw/stack3_20171023/stack/miniconda3-4.3.21-10a4fa6
SETUP_NDARRAY=ndarray 1.5.1.lsst1+5 -f Linux64 -Z /software/lsstsw/stack3_20171023/stack/miniconda3-4.3.21-10a4fa6
SETUP_NUMPY=numpy 1.13.1+2 -f Linux64 -Z /software/lsstsw/stack3_20171023/stack/miniconda3-4.3.21-10a4fa6
SETUP_OBS_BASE=obs_base 16.0-11-gfbb8ea7 -f Linux64 -Z /software/lsstsw/stack3_20171023/stack/miniconda3-4.3.21-10a4fa6
SETUP_OBS_CFHT=obs_cfht 16.0-2-gd82908a+30 -f Linux64 -Z /software/lsstsw/stack3_20171023/stack/miniconda3-4.3.21-10a4fa6
SETUP_OBS_COMCAM=obs_comCam 16.0-2-gea03357+1 -f Linux64 -Z /software/lsstsw/stack3_20171023/stack/miniconda3-4.3.21-10a4fa6
SETUP_OBS_CTIO0M9=obs_ctio0m9 16.0-1-g3247260+33 -f Linux64 -Z /software/lsstsw/stack3_20171023/stack/miniconda3-4.3.21-10a4fa6
SETUP_OBS_DECAM=obs_decam 16.0-4-gb463f2d+17 -f Linux64 -Z /software/lsstsw/stack3_20171023/stack/miniconda3-4.3.21-10a4fa6
SETUP_OBS_LSSTSIM=obs_lsstSim 16.0-4-g7986501+19 -f Linux64 -Z /software/lsstsw/stack3_20171023/stack/miniconda3-4.3.21-10a4fa6
SETUP_OBS_MONOCAM=obs_monocam 16.0-2-g379f3ce+11 -f Linux64 -Z /software/lsstsw/stack3_20171023/stack/miniconda3-4.3.21-10a4fa6
SETUP_OBS_SDSS=obs_sdss 16.0-3-g44ea68f+31 -f Linux64 -Z /software/lsstsw/stack3_20171023/stack/miniconda3-4.3.21-10a4fa6
SETUP_OBS_SUBARU=obs_subaru 16.0-20-gb58d072b -f Linux64 -Z /software/lsstsw/stack3_20171023/stack/miniconda3-4.3.21-10a4fa6
SETUP_OBS_TEST=obs_test 16.0-4-ge3254b7+8 -f Linux64 -Z /software/lsstsw/stack3_20171023/stack/miniconda3-4.3.21-10a4fa6
SETUP_PEP8_NAMING=pep8_naming 0.4.1+3 -f Linux64 -Z /software/lsstsw/stack3_20171023/stack/miniconda3-4.3.21-10a4fa6
SETUP_PEX_CONFIG=pex_config 16.0-3-g9645794+3 -f Linux64 -Z /software/lsstsw/stack3_20171023/stack/miniconda3-4.3.21-10a4fa6
SETUP_PEX_EXCEPTIONS=pex_exceptions 16.0-2-gf4e7cdd+3 -f Linux64 -Z /software/lsstsw/stack3_20171023/stack/miniconda3-4.3.21-10a4fa6
SETUP_PEX_POLICY=pex_policy 16.0-2-gf41ba6b+3 -f Linux64 -Z /software/lsstsw/stack3_20171023/stack/miniconda3-4.3.21-10a4fa6
SETUP_PIPE_BASE=pipe_base 16.0-11-g9fe0e56+7 -f Linux64 -Z /software/lsstsw/stack3_20171023/stack/miniconda3-4.3.21-10a4fa6
SETUP_PIPE_DRIVERS=pipe_drivers 16.0-4-g17aeff3 -f Linux64 -Z /software/lsstsw/stack3_20171023/stack/miniconda3-4.3.21-10a4fa6
SETUP_PIPE_TASKS=pipe_tasks 16.0-12-g85846a17 -f Linux64 -Z /software/lsstsw/stack3_20171023/stack/miniconda3-4.3.21-10a4fa6
SETUP_PROXMIN=proxmin lsst-dev-g87d7c0f1bb -f Linux64 -Z /software/lsstsw/stack3_20171023/stack/miniconda3-4.3.21-10a4fa6
SETUP_PSFEX=psfex 16.0+4 -f Linux64 -Z /software/lsstsw/stack3_20171023/stack/miniconda3-4.3.21-10a4fa6
SETUP_PYBIND11=pybind11 2.2.3.lsst1 -f Linux64 -Z /software/lsstsw/stack3_20171023/stack/miniconda3-4.3.21-10a4fa6
SETUP_PYCODESTYLE=pycodestyle 2.3.1+3 -f Linux64 -Z /software/lsstsw/stack3_20171023/stack/miniconda3-4.3.21-10a4fa6
SETUP_PYFLAKES=pyflakes 1.6.0+2 -f Linux64 -Z /software/lsstsw/stack3_20171023/stack/miniconda3-4.3.21-10a4fa6
SETUP_PYKG_CONFIG=pykg_config 1.3.0+4 -f Linux64 -Z /software/lsstsw/stack3_20171023/stack/miniconda3-4.3.21-10a4fa6
SETUP_PYTEST=pytest 3.2.0.lsst4+1 -f Linux64 -Z /software/lsstsw/stack3_20171023/stack/miniconda3-4.3.21-10a4fa6
SETUP_PYTEST_COV=pytest_cov 2.5.1 -f Linux64 -Z /software/lsstsw/stack3_20171023/stack/miniconda3-4.3.21-10a4fa6
SETUP_PYTEST_FLAKE8=pytest_flake8 0.9.1+8 -f Linux64 -Z /software/lsstsw/stack3_20171023/stack/miniconda3-4.3.21-10a4fa6
SETUP_PYTEST_FORKED=pytest_forked 0.2.lsst4+1 -f Linux64 -Z /software/lsstsw/stack3_20171023/stack/miniconda3-4.3.21-10a4fa6
SETUP_PYTEST_SESSION2FILE=pytest_session2file 0.1.9+9 -f Linux64 -Z /software/lsstsw/stack3_20171023/stack/miniconda3-4.3.21-10a4fa6
SETUP_PYTEST_XDIST=pytest_xdist 1.20.1.lsst4+1 -f Linux64 -Z /software/lsstsw/stack3_20171023/stack/miniconda3-4.3.21-10a4fa6
SETUP_PYTHON=python 0.0.8 -f Linux64 -Z /software/lsstsw/stack3_20171023/stack/miniconda3-4.3.21-10a4fa6
SETUP_PYTHON_COVERAGE=python_coverage 4.5.1 -f Linux64 -Z /software/lsstsw/stack3_20171023/stack/miniconda3-4.3.21-10a4fa6
SETUP_PYTHON_EXECNET=python_execnet 1.4.1.lsst4+1 -f Linux64 -Z /software/lsstsw/stack3_20171023/stack/miniconda3-4.3.21-10a4fa6
SETUP_PYTHON_FUTURE=python_future 0.16.0+3 -f Linux64 -Z /software/lsstsw/stack3_20171023/stack/miniconda3-4.3.21-10a4fa6
SETUP_PYTHON_MCCABE=python_mccabe 0.6.1+9 -f Linux64 -Z /software/lsstsw/stack3_20171023/stack/miniconda3-4.3.21-10a4fa6
SETUP_PYTHON_PSUTIL=python_psutil 5.4.3+2 -f Linux64 -Z /software/lsstsw/stack3_20171023/stack/miniconda3-4.3.21-10a4fa6
SETUP_PYYAML=pyyaml 3.13 -f Linux64 -Z /software/lsstsw/stack3_20171023/stack/miniconda3-4.3.21-10a4fa6
SETUP_REQUESTS=requests 2.9.1.lsst1+4 -f Linux64 -Z /software/lsstsw/stack3_20171023/stack/miniconda3-4.3.21-10a4fa6
SETUP_SCARLET=scarlet lsst-dev-gfdfbe1c240+1 -f Linux64 -Z /software/lsstsw/stack3_20171023/stack/miniconda3-4.3.21-10a4fa6
SETUP_SCIPY=scipy 0.0.1.lsst1+7 -f Linux64 -Z /software/lsstsw/stack3_20171023/stack/miniconda3-4.3.21-10a4fa6
SETUP_SCONS=scons 3.0.0.lsst1+3 -f Linux64 -Z /software/lsstsw/stack3_20171023/stack/miniconda3-4.3.21-10a4fa6
SETUP_SCONSUTILS=sconsUtils 16.0-5-ga6f992e -f Linux64 -Z /software/lsstsw/stack3_20171023/stack/miniconda3-4.3.21-10a4fa6
SETUP_SHAPELET=shapelet 16.0-6-gf0acd13+8 -f Linux64 -Z /software/lsstsw/stack3_20171023/stack/miniconda3-4.3.21-10a4fa6
SETUP_SKYMAP=skymap 16.0-3-g6923fb6+9 -f Linux64 -Z /software/lsstsw/stack3_20171023/stack/miniconda3-4.3.21-10a4fa6
SETUP_SPHGEOM=sphgeom 16.0-4-g5f3a788+5 -f Linux64 -Z /software/lsstsw/stack3_20171023/stack/miniconda3-4.3.21-10a4fa6
SETUP_SQLALCHEMY=sqlalchemy 1.2.2+2 -f Linux64 -Z /software/lsstsw/stack3_20171023/stack/miniconda3-4.3.21-10a4fa6
SETUP_STARLINK_AST=starlink_ast lsst-dev-gd6cc4e835a -f Linux64 -Z /software/lsstsw/stack3_20171023/stack/miniconda3-4.3.21-10a4fa6
SETUP_SYNPIPE=synpipe 16.0-4-g7690030+17 -f Linux64 -Z /software/lsstsw/stack3_20171023/stack/miniconda3-4.3.21-10a4fa6
SETUP_TMV=tmv 0.73.lsst2+2 -f Linux64 -Z /software/lsstsw/stack3_20171023/stack/miniconda3-4.3.21-10a4fa6
SETUP_TREECORR=treecorr 3.2.3.lsst3+3 -f Linux64 -Z /software/lsstsw/stack3_20171023/stack/miniconda3-4.3.21-10a4fa6
SETUP_UTILS=utils 16.0-6-g3610b4f+3 -f Linux64 -Z /software/lsstsw/stack3_20171023/stack/miniconda3-4.3.21-10a4fa6
SETUP_VALIDATE_DRP=validate_drp 16.0-9-gf51232b+6 -f Linux64 -Z /software/lsstsw/stack3_20171023/stack/miniconda3-4.3.21-10a4fa6
SETUP_VERIFY=verify 16.0-1-g0ee5b77+6 -f Linux64 -Z /software/lsstsw/stack3_20171023/stack/miniconda3-4.3.21-10a4fa6
SETUP_VERIFY_METRICS=verify_metrics 16.0-2-g16e3d1c+3 -f Linux64 -Z /software/lsstsw/stack3_20171023/stack/miniconda3-4.3.21-10a4fa6
SETUP_WCSLIB=wcslib 5.13.lsst1+2 -f Linux64 -Z /software/lsstsw/stack3_20171023/stack/miniconda3-4.3.21-10a4fa6
SETUP_WS4PY=ws4py 0.4.2+3 -f Linux64 -Z /software/lsstsw/stack3_20171023/stack/miniconda3-4.3.21-10a4fa6
SETUP_XPA=xpa 2.1.15.lsst3 -f Linux64 -Z /software/lsstsw/stack3_20171023/stack/miniconda3-4.3.21-10a4fa6
SHAPELET_DIR=/software/lsstsw/stack3_20171023/stack/miniconda3-4.3.21-10a4fa6/Linux64/shapelet/16.0-6-gf0acd13+8
SKYMAP_DIR=/software/lsstsw/stack3_20171023/stack/miniconda3-4.3.21-10a4fa6/Linux64/skymap/16.0-3-g6923fb6+9
SPHGEOM_DIR=/software/lsstsw/stack3_20171023/stack/miniconda3-4.3.21-10a4fa6/Linux64/sphgeom/16.0-4-g5f3a788+5
SQLALCHEMY_DIR=/software/lsstsw/stack3_20171023/stack/miniconda3-4.3.21-10a4fa6/Linux64/sqlalchemy/1.2.2+2
SQUASH_USER=nobody
SQUASH_password=
STARLINK_AST_DIR=/software/lsstsw/stack3_20171023/stack/miniconda3-4.3.21-10a4fa6/Linux64/starlink_ast/lsst-dev-gd6cc4e835a
SYNPIPE_DIR=/software/lsstsw/stack3_20171023/stack/miniconda3-4.3.21-10a4fa6/Linux64/synpipe/16.0-4-g7690030+17
TMV_DIR=/software/lsstsw/stack3_20171023/stack/miniconda3-4.3.21-10a4fa6/Linux64/tmv/0.73.lsst2+2
TREECORR_DIR=/software/lsstsw/stack3_20171023/stack/miniconda3-4.3.21-10a4fa6/Linux64/treecorr/3.2.3.lsst3+3
UTILS_DIR=/software/lsstsw/stack3_20171023/stack/miniconda3-4.3.21-10a4fa6/Linux64/utils/16.0-6-g3610b4f+3
VALIDATE_DRP_DIR=/software/lsstsw/stack3_20171023/stack/miniconda3-4.3.21-10a4fa6/Linux64/validate_drp/16.0-9-gf51232b+6
VERIFY_DIR=/software/lsstsw/stack3_20171023/stack/miniconda3-4.3.21-10a4fa6/Linux64/verify/16.0-1-g0ee5b77+6
VERIFY_METRICS_DIR=/software/lsstsw/stack3_20171023/stack/miniconda3-4.3.21-10a4fa6/Linux64/verify_metrics/16.0-2-g16e3d1c+3
VERSION_TAG=w_2018_34
WCSLIB_DIR=/software/lsstsw/stack3_20171023/stack/miniconda3-4.3.21-10a4fa6/Linux64/wcslib/5.13.lsst1+2
WS4PY_DIR=/software/lsstsw/stack3_20171023/stack/miniconda3-4.3.21-10a4fa6/Linux64/ws4py/0.4.2+3
XDG_RUNTIME_DIR=/run/user/34076
XDG_SESSION_ID=56101
XPA_DIR=/software/lsstsw/stack3_20171023/stack/miniconda3-4.3.21-10a4fa6/Linux64/xpa/2.1.15.lsst3
X_SCLS=devtoolset-6
\end{verbatim}


% Include all the relevant bib files.
% https://lsst-texmf.lsst.io/lsstdoc.html#bibliographies
\bibliography{lsst,lsst-dm,refs_ads,refs,books}

\end{document}
